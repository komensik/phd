% Options for packages loaded elsewhere
\PassOptionsToPackage{unicode}{hyperref}
\PassOptionsToPackage{hyphens}{url}
\PassOptionsToPackage{dvipsnames,svgnames,x11names}{xcolor}
%
\documentclass[
]{apsr}

\usepackage{amsmath,amssymb}
\usepackage{iftex}
\ifPDFTeX
  \usepackage[T1]{fontenc}
  \usepackage[utf8]{inputenc}
  \usepackage{textcomp} % provide euro and other symbols
\else % if luatex or xetex
  \usepackage{unicode-math}
  \defaultfontfeatures{Scale=MatchLowercase}
  \defaultfontfeatures[\rmfamily]{Ligatures=TeX,Scale=1}
\fi
\usepackage{lmodern}
\ifPDFTeX\else  
    % xetex/luatex font selection
\fi
% Use upquote if available, for straight quotes in verbatim environments
\IfFileExists{upquote.sty}{\usepackage{upquote}}{}
\IfFileExists{microtype.sty}{% use microtype if available
  \usepackage[]{microtype}
  \UseMicrotypeSet[protrusion]{basicmath} % disable protrusion for tt fonts
}{}
\makeatletter
\@ifundefined{KOMAClassName}{% if non-KOMA class
  \IfFileExists{parskip.sty}{%
    \usepackage{parskip}
  }{% else
    \setlength{\parindent}{0pt}
    \setlength{\parskip}{6pt plus 2pt minus 1pt}}
}{% if KOMA class
  \KOMAoptions{parskip=half}}
\makeatother
\usepackage{xcolor}
\ifLuaTeX
  \usepackage{luacolor}
  \usepackage[soul]{lua-ul}
\else
  \usepackage{soul}
  
\fi
\setlength{\emergencystretch}{3em} % prevent overfull lines
\setcounter{secnumdepth}{5}
% Make \paragraph and \subparagraph free-standing
\makeatletter
\ifx\paragraph\undefined\else
  \let\oldparagraph\paragraph
  \renewcommand{\paragraph}{
    \@ifstar
      \xxxParagraphStar
      \xxxParagraphNoStar
  }
  \newcommand{\xxxParagraphStar}[1]{\oldparagraph*{#1}\mbox{}}
  \newcommand{\xxxParagraphNoStar}[1]{\oldparagraph{#1}\mbox{}}
\fi
\ifx\subparagraph\undefined\else
  \let\oldsubparagraph\subparagraph
  \renewcommand{\subparagraph}{
    \@ifstar
      \xxxSubParagraphStar
      \xxxSubParagraphNoStar
  }
  \newcommand{\xxxSubParagraphStar}[1]{\oldsubparagraph*{#1}\mbox{}}
  \newcommand{\xxxSubParagraphNoStar}[1]{\oldsubparagraph{#1}\mbox{}}
\fi
\makeatother


\providecommand{\tightlist}{%
  \setlength{\itemsep}{0pt}\setlength{\parskip}{0pt}}\usepackage{longtable,booktabs,array}
\usepackage{calc} % for calculating minipage widths
% Correct order of tables after \paragraph or \subparagraph
\usepackage{etoolbox}
\makeatletter
\patchcmd\longtable{\par}{\if@noskipsec\mbox{}\fi\par}{}{}
\makeatother
% Allow footnotes in longtable head/foot
\IfFileExists{footnotehyper.sty}{\usepackage{footnotehyper}}{\usepackage{footnote}}
\makesavenoteenv{longtable}
\usepackage{graphicx}
\makeatletter
\def\maxwidth{\ifdim\Gin@nat@width>\linewidth\linewidth\else\Gin@nat@width\fi}
\def\maxheight{\ifdim\Gin@nat@height>\textheight\textheight\else\Gin@nat@height\fi}
\makeatother
% Scale images if necessary, so that they will not overflow the page
% margins by default, and it is still possible to overwrite the defaults
% using explicit options in \includegraphics[width, height, ...]{}
\setkeys{Gin}{width=\maxwidth,height=\maxheight,keepaspectratio}
% Set default figure placement to htbp
\makeatletter
\def\fps@figure{htbp}
\makeatother

% Set double spacing (apsr.cls already does this, but okay to reinforce)
\usepackage{setspace}
\doublespacing

% Use Times-like font
\usepackage{mathptmx}

% Optional: cleaner figure/table spacing
\usepackage{caption}
\captionsetup{font=small, labelfont=bf}

% Optional: remove paragraph indentation
\setlength\parindent{0pt}
\setlength\parskip{6pt}

% Optional: cleaner headers
\usepackage{sectsty}
\sectionfont{\normalsize\bfseries}
\subsectionfont{\normalsize\itshape}
\makeatletter
\@ifpackageloaded{caption}{}{\usepackage{caption}}
\AtBeginDocument{%
\ifdefined\contentsname
  \renewcommand*\contentsname{Table of contents}
\else
  \newcommand\contentsname{Table of contents}
\fi
\ifdefined\listfigurename
  \renewcommand*\listfigurename{List of Figures}
\else
  \newcommand\listfigurename{List of Figures}
\fi
\ifdefined\listtablename
  \renewcommand*\listtablename{List of Tables}
\else
  \newcommand\listtablename{List of Tables}
\fi
\ifdefined\figurename
  \renewcommand*\figurename{Figure}
\else
  \newcommand\figurename{Figure}
\fi
\ifdefined\tablename
  \renewcommand*\tablename{Table}
\else
  \newcommand\tablename{Table}
\fi
}
\@ifpackageloaded{float}{}{\usepackage{float}}
\floatstyle{ruled}
\@ifundefined{c@chapter}{\newfloat{codelisting}{h}{lop}}{\newfloat{codelisting}{h}{lop}[chapter]}
\floatname{codelisting}{Listing}
\newcommand*\listoflistings{\listof{codelisting}{List of Listings}}
\makeatother
\makeatletter
\makeatother
\makeatletter
\@ifpackageloaded{caption}{}{\usepackage{caption}}
\@ifpackageloaded{subcaption}{}{\usepackage{subcaption}}
\makeatother

\ifLuaTeX
  \usepackage{selnolig}  % disable illegal ligatures
\fi
\usepackage[]{natbib}
\bibliographystyle{plainnat}
\usepackage{bookmark}

\IfFileExists{xurl.sty}{\usepackage{xurl}}{} % add URL line breaks if available
\urlstyle{same} % disable monospaced font for URLs
\hypersetup{
  pdftitle={Prelim 3-23},
  pdfauthor={Kristina Mensik},
  colorlinks=true,
  linkcolor={blue},
  filecolor={Maroon},
  citecolor={Blue},
  urlcolor={Blue},
  pdfcreator={LaTeX via pandoc}}


\title{Prelim 3-23}
\author{Kristina Mensik}
\date{}

\begin{document}
\maketitle

\renewcommand*\contentsname{Table of contents}
{
\hypersetup{linkcolor=}
\setcounter{tocdepth}{3}
\tableofcontents
}

Revised intro from Adrianne:

Contact with the the state's carceral policy in the United States is
extensive. At year end 2001, more than 5.6 million U.S. adults had
served time in a state or federal prison (BOP 2003). Many millions more
experience incarceration through proximal contact -- because they have a
family member or friend who is incarcerated, or because they were the
victim of a crime prosecuted by the state. This direct and proximal
carceral contact has generated a significant scholarship in recent years
that attempts to understand how contact with this particular state
policy feeds back to shape attitudes towards the state (\emph{cite}),
attitudes towards policies (\emph{cite}), and political engagement
(\emph{cite}).

Implicit in much of this carceral policy feedback literature is the
contention that attitudes and engagement are shaped by perceptions of
what the targets of state policy \emph{deserve}. Across numerous policy
domains, individuals report being more willing to support policy when
the targets of that policy are viewed as deserving (cite). Yet, there is
limited evidence on how carceral contact, specifically, shapes
deservingness.

This paper addresses these two gaps\ldots To do so, I ask\ldots I
theorize that\ldots I collect new data\ldots{}

sA key insight from policy feedback literature is that the American
carceral state shapes political behavior. Policing and incarceration can
stymie political engagement, foment government distrust, and alter
narratives about one's own civic standing (Lerman and Weaver 2010, Owens
2014, Burch 2011, 2014, White 2019, Walker 2020). ~While this literature
deals with impacts on political mobilization -- particularly voting --
it has not addressed how contact with the criminal legal system shapes
policy opinions themselves. The closely interrelated literature on
social constructions holds that such opinions are based on constructions
about the given group targeted by policy, and specifically whether these
constructs are considered ``deserving of help, sympathy, and support''
(). While this literature has established that institutions and policies
shape perceptions of what targets of policy deserve, both it and policy
feedback scholarship leaves open the question of how contact with the
criminal legal shapes perceptions of what its \emph{targets} deserve.~
Further, both literatures have focused on interactions with the criminal
legal system of the ``accused'' -- policing and incarceration -- but
leave open the question of feedbacks experienced and perceptions of
target deservingness held by those who interact with the system as
victims.

In the criminal legal domain, perceptions of deservingness have myriad
real-world consequences. They may inform judgements about criminal
culpability in juries and sentencing decisions by judges, employers'
judgements about whether to hire returning citizens, or how a
neighborhood association responds to a proposal to construct a re-entry
home nearby. Literature on social constructions, however, situates
perceptions of deservingness as particularly important in their capacity
to shape policy preferences and ultimately determine policy outcomes
(Kreitzer and Smith 2018, Kreitzer, Maltby, and Smith 2022). However,
scant literature has tested this link in the criminal legal domain.
While scholars have begun to establish that experience with a policy can
generate warmer perceptions of that policy's targets, more recent
evidence suggests that judgements about what an individual deserves do
not always translate to support for policy that would accomplish that
end (Sorelle and Laws, 2022).

In this paper, \ul{I ask how experiences of incarceration and
victimization shape perceptions of what incarcerated people deserve, and
whether these perceptions translate to opinions on prison policy.} In
doing so, I seek to connect literatures on policy feedback and social
constructions of deservingness. To do so, I leverage novel survey data
to establish correlational relationship between incarceration and
victimization and perceptions of carceral citizens' deservingness.

My theoretical starting point is that perceptions of what groups like
carceral citizens deserve are often based on discursive social
constructions about them (Schneider and Ingram, 1993). Policy directly
shapes these constructions, as it produces both resource and
interpretive feedback effects that are part of the discourse that shapes
constructions. In forming opinions about policy and deciding what
inequalities are tolerable, publics draw on salient social constructions
about the relevant policy target and use them in a heuristic process to
quickly determine whether a target is ``deserving'' of help (citation).
Theoretically, then these judgements translate to, or at play in the
formation of, beliefs about criminal legal policy.

Here, I suggest that this loop between policy design and the messages it
transmits about what it's targets deserve may be interrupted for some
citizens due to their connections to those who are justice involved,
Emerging evidence that \emph{responses to} the criminal legal system's
policy messages vary, and specifically that some respond to the negative
messages transmitted by criminal legal policy with dubiousness towards
state legitimacy suggest perceptions of what its targets deserve may
vary as well (Walker 2020, Harris 2025).

. . .

\section{Literature}\label{literature}

How does contact with the criminal legal system affect perceptions of
it's targets deserve? And what do these deservingness perceptions mean
for policy views? These questions intersect with two literatures. First,
scholarship on social constructions and deservingness examine how groups
become characterized as deserving or undeserving of government
assistance, and theorizes that policy views hinge on such
characterizations (Schneider and Ingram 1993, Smith and Kreitzer 2024,
Oorschot 2006). While this literature leaves open the question of how
experiences with a policy might alter characterizations about it's
targets, policy feedback literature informs that gap by taking up
questions about how policies shape political attitudes and behavior.
{[}\emph{While policy feedback research has established that carceral
contact influences political participation and institutional trust, much
less is known about how such contact affects attitudes toward the
targets of these policies---specifically, how direct or proximal
experience with incarceration or victimization might reshape perceptions
of incarcerated people's deservingness and, in turn, opinions on prison
policy.{]}} I briefly review these literatures here.

In the wake of Schneider and Ingram's (1993) theory of social
constructions, a literature has coalesced around the idea that attitudes
towards policy targets are in dialectic with policy. That is, policy
design not only has ``interpretive effects'' on its targets but also
shapes discourse \emph{about} them. In this way, policy can ``telegraph
to the public how target groups should be treated,'' and helps to
produce social constructions or ``value-laden components, including
stereotypes, dominant ideologies, and assumptions'' about groups (Smith
and Kreitzer 2024, PAGE). As consequence, social constructions are
contingent -- for example, on local, state, and political and cultural
contexts and also the identity and ideology held by the `perceiver'
(Smith and Kreitzer 2018). This means both that policy can
\emph{reflect} existing social constructions, and feed back to causally
affect further social constructions of who targets are. The notion of
who targets \emph{are} in turn affects how people perceive what they
\emph{should} receive -- i.e., what they deserve.

A related stream of literature explains that people form quick
judgements about policy targets' deservingness by evaluating
constructions along five dimensions (``CARIN'') in an
evolutionarily-grounded impulse to help ``reciprocators'' and avoid
``free-riders'' (Petersen 2012, Aaroe and Petersen 2014). These criteria
primarily distinguish between recipients who cannot help their
circumstances versus those seeking unearned benefits, but refer to:
\emph{control,} which in its original conception was closely related to
\emph{need}, referring to whether someone is perceived as lazy or
unlucky -- a consideration Oorschot (2006) argues can override all
others. More recently, perception of a group or individual's \emph{need}
for assistance has considered the severity of their neediness and
\emph{control} the degree to which they are responsible (SoRelle and
Laws 2024). Perceptions of the group's \emph{attitude} refer to whether
the beneficiary is grateful and likeable, and closely related
perceptions of the group's \emph{reciprocity} connote their history or
apparent willingness to contribute to society. Last, the \emph{identity}
characteristic holds that people will evaluate more generously the
deservingness of people whose identity is aligned with their own
(Oorschot 2000, 2006).

Scholars have built on both social construction and CARIN theories of
deservingness to find that experience with a policy can influence how
targets of that policy are perceived. In the context of debt relief,
SoRelle and Laws (2023) find that people who have had student debt view
other borrowers as more deserving of debt forgiveness, and are more
supportive of government debt relief. Maltby and Kreitzer (2022) find
evidence that solidaristic effects may {[}begin to erode, may not be
generalists{]}. Specifically, they find that respondents who have been
on welfare view recipients of the same program as more deserving than
recipients of others --- a finding that may be explained through the
CARIN framework as the result of having more closely aligned identities,
or explained alternatively by the `stickiness' of negative social
constructions about welfare recipients. Findings from Soss's (2005) a
study of recipients of means tested versus non-means tested welfare,
supports the latter.

undeserving messages and resources effects unique to AFDC design, like
stringent behavioral monitoring and caseworker discretion in lieu of
standardized rules, lead its recipients to not only~ internalize a more
``undeserving'' perception of AFDC beneficiaries, but do so to the point
they seek to differentiate themselves from other beneficiaries and
splinter any sense of beneficiary group solidarity -- and splintering
their welfare policy views from their views of the deservingness of
beneficiaries.

\section{Theory}\label{theory}

How does being directly or proximally impacted by the criminal legal
system shape perceptions of what carceral citizens -- incarcerated
people, particularly -- deserve?

My theoretical starting point is that perceptions of what groups like
carceral citizens deserve are often shaped by social constructions about
them (Schneider and Ingram 1993). Like stereotypes, these constructions
are produced by policy and discourse -- both political, cultural, and
popular. On the individual level, then, these constructions are often
taken up in a heuristic process to form quick judgements about what a
target group deserves. Thus, predominant racialized constructions of
incarcerated people -- ``thugs,'' ``felons,'' or ``inmates'' --
suggesting culpable, ungrateful, ``takers'' responsible for their
condition of confinement are likely interpolated as undeserving,
translating to support for punitive carceral policy.~

Building on a policy feedback literature that shows interactions with
the criminal legal system shape political behavior and attitudes towards
the state (Anoll et al 2022, Anoll and Israel-Trummel 2019, Walker 2020,
Lerman and Weaver 2014), I propose interactions with the criminal legal
system also shape attitudes towards \emph{targets} of the state --
incarcerated people, what they deserve, and policy. ~

While some work finds that targets of policy can internalize negative or
undeserving constructions (Soss 1999), much of the work on criminal
legal contact stresses that demobilization is a product of resource
effects or interpretive effects \emph{distinct} from an ``undeserving''
self-conception (Lerman and Weaver 2014, White 2019). Instead, other
work emphasizes that direct contact with the criminal legal system can
heighten in-group solidarity (Lerman and Weaver 2014) and mobilize
despite resource and interpretive feedbacks that should be demobilizing,
specifically when policy targets reject policy ``teachings'' about their
lower civic status as part of a sense of systemic injustice --
suggesting a perception that the criminal legal system is less
legitimate or authoritative, and of its targets as distinctly deserving
(Walker 2020).

Proximal contact also generates political mobilization (Anoll et al
2022, White 2019b). Anoll's (2022) findings that family of the
incarcerated vote at higher rates in states with more stringent carceral
disenfranchisement policies suggests that ``surrogate'' mobilization is
motivated by a desire to act on behalf a ``deserving'' incarcerated
citizen. Further, from Anoll and Israel-Trummel's (2019) findings that
neighborhood exposure to racially discriminatory policing generate
perceptions of institutional \emph{il}legitimacy -- distrust in police
and government -- one might infer that exposure should also alter and
perhaps enhance perceptions in institutional targets as well.~

Coupled with deservingness scholarship on experience (Sorelle and Laws),
I expect that the perceptions of what carceral citizens deserve may look
different for people who have themselves been or who know carceral
citizens. Specifically, I hypothesize that \textbf{(H1) those with
direct or proximal contact, or who have close social ties to
incarcerated people, will view incarcerated people as more deserving}.
While Maltby and Kreitzer (2022) find that proximal contact does
\emph{not} impact perceptions of deservingness, there are important
limitations in their study that necessitate further examination of
direct and proximate impact. First, they report a dependent variable of
aggregated deservingness scores for disparate groups -- ranging from
``prisoners'' to ``opioid users'' to ``welfare cheats.'' Second, they
measure criminal legal contact by asking participants whether they or
someone they know has been either arrested or incarcerated. Grouping
these distinct experiences together muddies the waters. Arrest and
incarceration impact different swaths of the population (CITATION).
\hyperref[_msocom_1]{{[}KM1{]}}~Arrest alone is distinct in its
distribution of resources and political incentives as incarceration
(CITATION). And while short stints in jail can reduce voter
participation, individuals who are convicted, sentenced, and
incarcerated in prison encounter entirely different policy designs,
which ``teach'' very different ``lessons'' about one's civic standing
and government responsiveness. Soss (2005) finds that not only do
recipients of non-means tested welfare differentiate themselves as more
deserving of help than recipients of means-tested healthcare, but that
deserving-skeptic messages imbued in policy design (drug testing, a
requirement to prove need or exhaust alternative sources of aid) are so
pervasive as to motivate beneficiaries to maintain negative social
constructions of welfare recipients, from whom they try to differentiate
themselves and their networks. Thus, like affluent individuals for whom
inequality motivates a belief in meritocracy (Newman et al 2014),
individuals who are arrested and not found guilty, or are given
alternative ``lighter'' sentences than incarceration, may differentiate
themselves or their networks.

Most obviously, under a judicial system that maintains the presumption
of innocence and that leverages sentences based on seriousness,
individuals who are arrested and individuals who are incarcerated should
send very different messages about targets' deservingness to mass
publics and policymakers.

Placing carceral contact in the frame of cumulative punishment in the US
highlights another layer around which literature on policy feedbacks of
the carceral state has under-explored -- namely, the overlap of
``victims'' and ``perpetrators'' and policy feedbacks resulting from
contact with the criminal legal system as a victim of crime. To consider
how serious victimization may shape perception of target deservingness
and prison policy, I first note the clear and expected consequences of
victimization for these matters. Victimization is obviously a salient
experience that often generates anger (Ditton et al 1999) and drive
negative affect and perceptions of what criminals deserve. Victims of
crime, to be sure, usually engage the criminal legal system because they
believe their offender is deserving of punishment, and they are
deserving of establishing a sense of safety and justice. But it is also
true that to accomplish the latter, victims of violent crime have few
other options -- including if they know their perpetrator personally or
understand their perpetrator to be in need of help. Evidence that crime
victimization increases political participation (Bateson 2012),
including in resource-intensive activities like sustained community
organizing may offer pathways for coping with trauma (Rozowsky 2002,
Walklate 2007), but may also reflect more nuanced views of policy
changes needed to address root causes of crime.

Further, victims of violent crime are disproportionately part of
communities most heavily policed and incarcerated, and most victims of
violent crime know their offender (Bureau of Justice 2024). An extensive
literature shows that people who are incarcerated have
disproportionately been victims of crime -- one review article cited 31
of 37 papers support this overlap (Jennings et al 2012). While 2\% of
the general US population report being victims of violent crime, up to
45\% of carceral citizens have experienced pre-incarceration physical
abuse, and 8.5 to 39.2\% of specifically sexual abuse (Azimi et al 2019,
Carlson and Shafer 2010, Messina et al 2007, Wolff and Shi 2012, Yoder
et al.~2017). Still more experience violence while incarcerated (Wolff
et al 2009). ~Thus, while for many victims the proximity and salience of
a criminal may bolster negative social constructions and deservingness
perceptions, it might also translate to more nuanced perceptions of
culpability and a closer alignment of identity that moderates negative
affect.

While victims are usually presented as supportive of punitive policy in
public theater -- the recent naming of the Laken Riley Act, offers a
clear example -- the relationship is far less clear. Instead,
victimization does not appear to increase punitiveness (Kleck and
Jackson 2016), and victims instead appear more concerned with improved
victims compensation (Shapland 1984) and policies addressing ``root
causes'' of crime (ASJ 2024). Further, victims themselves may experience
a wide arrange of policy feedbacks, complicating views on policy
compared to those without direct experience. Thus, I include victims in
H1.

Social construction theory holds that perceptions of target group
deservingness translate to policy: legislators understand how groups are
perceived, and produce policy that, in this case, will punish
undeserving incarcerated targets. Deservingness scholars expect that
perceptions of deservingness translate directly into support policy
opinion, too. However, findings from Sorelle and Laws () and legislative
responsiveness literature () suggest neither relationship may be as
straightforward -- further still, literature in the criminal legal
domain has not assessed this relationship directly. In the criminal
legal domain, I expect that the salience of negative social
constructions of policy targets could make perceptions of deservingness
``stickier'' than views on what policy help or punishment targets
deserve. Thus, while I expect that individuals who are directly or
proximately impacted by the criminal legal system will hold different
beliefs about criminal policy than those who are not, the role of
deservingness perceptions in this context remains to be seen.

To summarize:

H1: People who are directly or proximally impacted by the criminal legal
system will see incarcerated people as more deserving than people who do
not.

1B: Among those who xyz,

H2:~ People who are directly or proximally impacted by the criminal
legal system will hold different beliefs about criminal legal policy
\ul{than those who are not}.~

*Relationship between deservingness and policy preferences may look
different for those who are impacted than those who are not impacted.

~\hyperref[_msoanchor_1]{{[}KM1{]}}Move to methods? Or, because their
findings contradict the H, I thought I should still address somewhat
here?

. . .

\section{Data and Methods}\label{data-and-methods}

\textbf{Analysis}

I test these hypotheses using a nationally-representative survey that
asks about participants' direct and proximal experience with
incarceration or victimization of violent crime, perceptions of
deservingness, and views on policy. In this section, I describe these
data and methods for analyses.

\textbf{Data}

\emph{Survey}

I explore the application of deservingness to carceral contexts using
novel survey data collected in September 2024 by Smith, Bowman, and
Mensik.~ Kreitzer and Smith, previously empirically mapped constructions
of power and deservingness of 87 target populations by employing MTurk
workers; this new data set mimics the strengths of that data collection
strategy, and makes important changes, such as relying on survey
respondents recruited from Bovitz proprietary panel and reducing the
number of groups individuals assess, to address issues of respondent
fatigue.

This survey targets the US adult general population. 2,716 participants
were on recruited to complete surveys on the platform Forthright. To be
eligible for the study, participants were required to provide informed
consent, be at least 18 years old, and be residents of the US. They were
paid \$10 for completing the 30-minute survey. The sample is just under
51\% women, and 49\% men. The sample is more Democratic than the
national average (39\% versus 31\%), but as Republican (25\%). I apply
exclusions for low response quality failing to pass two attention
checks, satisficing, and extreme outlier response
times\hyperref[_msocom_1]{{[}KM1{]}}~.

\emph{Measuring Deservingness}

My first key variable of interest is perceptions of target group
deservingness. These perceptions are measured as in Kreitzer and Smith
(2018) . Participants are asked to consider a definitional deservingness
prompt, and indicate their perception of group deservingness using a
0-100 slider scale set to 50. Participants rate the deseringness of a
total of 65 groups, including ``incarcerated people/prisoners.''

I standardize deservingness by taking difference between the score a
respondent gives to a given group and the average score the respondent
gave across all groups. I also normalize deservingness scores by
calculating mean deservingness scores of incarcerated people for all
respondents and subtract each respondent's individual rating of
incarcerated people from this mean to report whether incarcerated
deservingness ratings are above or below average.

As I describe in the theory section, I expect that experiencing
incarceration, or having close ties with someone who has, and
experiences of victimization affect perceptions of target deservingness
and policy preferences. To measure contact and proximate contact, I ask
respondents whether they or someone they know well have been
``incarcerated in jail or prison,'' and/or whether they ``been a victim
of violent crime.'' I operationalize these measures using three separate
dummy variables.

I include variables to control for alternative explanations for both how
policy exposure might impact perceptions of target deservingness.

Perceptions of an incarcerated person's deservingness may reflect
broader attitudes towards policy targets who are implicitly constructed
as needy and or deviant. \emph{This may be particularly true in this
experiment design, where participants are given definitions of deserving
(groups ``contributing to the general welfare of society and worthy, and
thus\ldots deserving of sympathy, pity, or help {[}and{]} \ldots good,
smart, hardworking, loyal, disciplined, generous, caring of others,
respectful, and creative'') and undeserving (``burden to the general
welfare of society, and believed to be undeserving of sympathy, pity, or
help,'' and described as ``greedy, disrespectful, disloyal, immoral,
disgusting, dangerous, lazy, and expect{[}ing{]} others to care for
them.''}). To control for this general tendency, I compile deservingness
scores of groups who fit these frames, such as different welfare
recipients, Medicaid and Medicare/SSN recipients, the unemployed, poor
families, homeless individuals, and asylum seekers/refugees.

\ldots{} these perceptions could reasonably reflect negative attitudes
towards and beliefs about people who are presumed to commit crimes.
``Punitiveness,'' broadly conceived as tendencies to explain crime
through the lens of individual culpability and moral failing, and
prioritize retributive justice in policy responses to crime, may
definitially prescribe perceptions that people who are incarcerated are
\emph{undeserving}.

\begin{itemize}
\tightlist
\item
  note about conditions
\end{itemize}

anchor support for punitive (rather than rehabilitative) policy in a
retributive framework

In the US, the ``punitive impulse'' or tendency to explain crime through
the lens of individual culpability and support retributive justice is

Public opinion and criminological literature dealing with attitudes
towards criminal legal policy usually, and rightly, control for
punitiveness. I

. . .

\section{Conclusion}\label{conclusion}

. . .


\renewcommand\refname{References}
  \bibliography{references.bib}



\end{document}

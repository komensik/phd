% Options for packages loaded elsewhere
\PassOptionsToPackage{unicode}{hyperref}
\PassOptionsToPackage{hyphens}{url}
\PassOptionsToPackage{dvipsnames,svgnames,x11names}{xcolor}
%
\documentclass[
  letterpaper,
  DIV=11,
  numbers=noendperiod]{scrartcl}

\usepackage{amsmath,amssymb}
\usepackage{iftex}
\ifPDFTeX
  \usepackage[T1]{fontenc}
  \usepackage[utf8]{inputenc}
  \usepackage{textcomp} % provide euro and other symbols
\else % if luatex or xetex
  \usepackage{unicode-math}
  \defaultfontfeatures{Scale=MatchLowercase}
  \defaultfontfeatures[\rmfamily]{Ligatures=TeX,Scale=1}
\fi
\usepackage{lmodern}
\ifPDFTeX\else  
    % xetex/luatex font selection
\fi
% Use upquote if available, for straight quotes in verbatim environments
\IfFileExists{upquote.sty}{\usepackage{upquote}}{}
\IfFileExists{microtype.sty}{% use microtype if available
  \usepackage[]{microtype}
  \UseMicrotypeSet[protrusion]{basicmath} % disable protrusion for tt fonts
}{}
\makeatletter
\@ifundefined{KOMAClassName}{% if non-KOMA class
  \IfFileExists{parskip.sty}{%
    \usepackage{parskip}
  }{% else
    \setlength{\parindent}{0pt}
    \setlength{\parskip}{6pt plus 2pt minus 1pt}}
}{% if KOMA class
  \KOMAoptions{parskip=half}}
\makeatother
\usepackage{xcolor}
\setlength{\emergencystretch}{3em} % prevent overfull lines
\setcounter{secnumdepth}{-\maxdimen} % remove section numbering
% Make \paragraph and \subparagraph free-standing
\ifx\paragraph\undefined\else
  \let\oldparagraph\paragraph
  \renewcommand{\paragraph}[1]{\oldparagraph{#1}\mbox{}}
\fi
\ifx\subparagraph\undefined\else
  \let\oldsubparagraph\subparagraph
  \renewcommand{\subparagraph}[1]{\oldsubparagraph{#1}\mbox{}}
\fi


\providecommand{\tightlist}{%
  \setlength{\itemsep}{0pt}\setlength{\parskip}{0pt}}\usepackage{longtable,booktabs,array}
\usepackage{calc} % for calculating minipage widths
% Correct order of tables after \paragraph or \subparagraph
\usepackage{etoolbox}
\makeatletter
\patchcmd\longtable{\par}{\if@noskipsec\mbox{}\fi\par}{}{}
\makeatother
% Allow footnotes in longtable head/foot
\IfFileExists{footnotehyper.sty}{\usepackage{footnotehyper}}{\usepackage{footnote}}
\makesavenoteenv{longtable}
\usepackage{graphicx}
\makeatletter
\def\maxwidth{\ifdim\Gin@nat@width>\linewidth\linewidth\else\Gin@nat@width\fi}
\def\maxheight{\ifdim\Gin@nat@height>\textheight\textheight\else\Gin@nat@height\fi}
\makeatother
% Scale images if necessary, so that they will not overflow the page
% margins by default, and it is still possible to overwrite the defaults
% using explicit options in \includegraphics[width, height, ...]{}
\setkeys{Gin}{width=\maxwidth,height=\maxheight,keepaspectratio}
% Set default figure placement to htbp
\makeatletter
\def\fps@figure{htbp}
\makeatother
% definitions for citeproc citations
\NewDocumentCommand\citeproctext{}{}
\NewDocumentCommand\citeproc{mm}{%
  \begingroup\def\citeproctext{#2}\cite{#1}\endgroup}
\makeatletter
 % allow citations to break across lines
 \let\@cite@ofmt\@firstofone
 % avoid brackets around text for \cite:
 \def\@biblabel#1{}
 \def\@cite#1#2{{#1\if@tempswa , #2\fi}}
\makeatother
\newlength{\cslhangindent}
\setlength{\cslhangindent}{1.5em}
\newlength{\csllabelwidth}
\setlength{\csllabelwidth}{3em}
\newenvironment{CSLReferences}[2] % #1 hanging-indent, #2 entry-spacing
 {\begin{list}{}{%
  \setlength{\itemindent}{0pt}
  \setlength{\leftmargin}{0pt}
  \setlength{\parsep}{0pt}
  % turn on hanging indent if param 1 is 1
  \ifodd #1
   \setlength{\leftmargin}{\cslhangindent}
   \setlength{\itemindent}{-1\cslhangindent}
  \fi
  % set entry spacing
  \setlength{\itemsep}{#2\baselineskip}}}
 {\end{list}}
\usepackage{calc}
\newcommand{\CSLBlock}[1]{\hfill\break\parbox[t]{\linewidth}{\strut\ignorespaces#1\strut}}
\newcommand{\CSLLeftMargin}[1]{\parbox[t]{\csllabelwidth}{\strut#1\strut}}
\newcommand{\CSLRightInline}[1]{\parbox[t]{\linewidth - \csllabelwidth}{\strut#1\strut}}
\newcommand{\CSLIndent}[1]{\hspace{\cslhangindent}#1}

\usepackage{setspace}
\doublespacing

\usepackage{mathpazo}
\usepackage{amsmath}
\usepackage{titling}
\usepackage{etoolbox}
\usepackage{sectsty}
\usepackage{caption}
\usepackage{tcolorbox}
\usepackage{endnotes}

% --- Title and Author formatting ---
\pretitle{\begin{center}\LARGE\bfseries}
\posttitle{\par\end{center}\vskip 1em}
\preauthor{\begin{center}\large}
\postauthor{\par\end{center}\vskip 2em}
\predate{}
\postdate{}

% --- Page setup ---
\setlength{\droptitle}{6em}
\setlength\parindent{0pt}
\setlength\parskip{6pt}
\allsectionsfont{\rmfamily\bfseries\normalsize}
\captionsetup{font=small, labelfont=bf}
\tcbset{
  colback=gray!10!white,
  colframe=gray!50!black,
  boxrule=0.5pt, arc=2pt,
  left=6pt, right=6pt, top=6pt, bottom=6pt
}
\let\footnote=\endnote

% --- Put Introduction on new page ---
\pretocmd{\section}{\clearpage}{}{}
\KOMAoption{captions}{tableheading,figureheading}
\makeatletter
\@ifpackageloaded{caption}{}{\usepackage{caption}}
\AtBeginDocument{%
\ifdefined\contentsname
  \renewcommand*\contentsname{Table of contents}
\else
  \newcommand\contentsname{Table of contents}
\fi
\ifdefined\listfigurename
  \renewcommand*\listfigurename{List of Figures}
\else
  \newcommand\listfigurename{List of Figures}
\fi
\ifdefined\listtablename
  \renewcommand*\listtablename{List of Tables}
\else
  \newcommand\listtablename{List of Tables}
\fi
\ifdefined\figurename
  \renewcommand*\figurename{Figure}
\else
  \newcommand\figurename{Figure}
\fi
\ifdefined\tablename
  \renewcommand*\tablename{Table}
\else
  \newcommand\tablename{Table}
\fi
}
\@ifpackageloaded{float}{}{\usepackage{float}}
\floatstyle{ruled}
\@ifundefined{c@chapter}{\newfloat{codelisting}{h}{lop}}{\newfloat{codelisting}{h}{lop}[chapter]}
\floatname{codelisting}{Listing}
\newcommand*\listoflistings{\listof{codelisting}{List of Listings}}
\makeatother
\makeatletter
\makeatother
\makeatletter
\@ifpackageloaded{caption}{}{\usepackage{caption}}
\@ifpackageloaded{subcaption}{}{\usepackage{subcaption}}
\makeatother
\ifLuaTeX
  \usepackage{selnolig}  % disable illegal ligatures
\fi
\usepackage{bookmark}

\IfFileExists{xurl.sty}{\usepackage{xurl}}{} % add URL line breaks if available
\urlstyle{same} % disable monospaced font for URLs
\hypersetup{
  pdftitle={The Politics of Proximity: Attitudinal Feedback Effects of the Carceral State},
  pdfauthor={Kristina Mensik Department of Political Science, Duke University},
  colorlinks=true,
  linkcolor={blue},
  filecolor={Maroon},
  citecolor={Blue},
  urlcolor={Blue},
  pdfcreator={LaTeX via pandoc}}

\title{The Politics of Proximity: Attitudinal Feedback Effects of the
Carceral State}
\author{Kristina Mensik Department of Political Science, Duke
University}
\date{}

\begin{document}
\maketitle
\begin{abstract}
This paper examines how personal and proximate contact with the criminal
legal system shapes public perceptions of deservingness and support for
related policies. Drawing on original survey data from 2,716 U.S.
adults, I test whether individuals with lived or vicarious carceral
experiences, including those victimized by a crime prosecuted by the
state, view incarcerated people as more deserving of help, and whether
these perceptions mediate support for decarceral or punitive policies.
\end{abstract}

\newpage

\subsection{Introduction}\label{introduction}

Personal experience with carceral policy in the United States is
extensive. By year-end 2001, more than 5.6 million U.S. adults had
served time in a state or federal prison (BOP 2003), and by 2010, an
estimated 8 \% had a felony conviction and 3 \% had served or were
serving prison time (Shannon et al 2017). While systematized data on
non-felony convictions remain, as Stenson and Mayson (2018) put it,
``absurdly, embarassingly'' (p.~732) elusive, they estimate 13.2 million
such cases are filed annually. Many millions more experience
incarceration indirectly -- through proximal contact -- as family,
friends, or neighbors of those incarcerated -- or through victimization,
having been harmed by a crime prosecuted by the state. While the
interests of these two groups are regularly presented as oppositional,
we actually know very little about how these experiences --
incarceration, proximal carceral contact, and victimization -- shape how
people think about criminal legal policy.

Scholars have shown that an important driver of public support for state
intervention across other domains is people's beliefs about whether a
given target group is \emph{deserving} of policy help or punishment
(DeSante 2013; Ellis and Faricy 2020). Further still, such deservingness
perceptions often predict state policy outcomes (Kreitzer, Maltby, and
Smith 2022a). These perceptions of deservingness are not purely personal
-- they are shaped by myriad factors including policy designs themselves
(A. L. Schneider and Ingram 2019). Yet, while scholars have shown that
deservingness perceptions inform opinions about social welfare,
immigration, and even gun policy (W. van Oorschot 2000a; Kreitzer,
Maltby, and Smith 2022b), the role of deservingness in the criminal
legal context remains less well-developed. In particular, we do not know
how people's experiences with the carceral state---either through direct
or proximal contact---shape perceptions of deservingness when it comes
to criminal justice policy.

Recent work (SoRelle and Laws 2024a) suggests that experience with
policy may condition the logics people use to assess deservingness --
and that the role of personal experience in deservingness remains
underexplored. At the same time, a growing body of scholarship
demonstrates how contact with the carceral state shapes people's
political attitudes (Weaver and Lerman 2010a), political identities
(Owens 2014), and political engagement (White 2019). Through what these
scholars term policy feedback effects, policies, once enacted, reshape
politics by redistributing resources and sending interpretive signals
that affect how individuals understand their -- and other's --
relationship to the state (Skocpol 1992; Pierson 1993; Weible 2023).

This paper brings deservingness and policy feedback frameworks together
to understand how direct and proximate experiences of incarceration and
victimization shape perceptions of the deservingness of incarcerated
people, and whether these perceptions translate to opinions on prison
policy. I start with the contention that people form views about policy
by quickly judging social constructions of that policy's target group
(A. Schneider and Ingram 1993a). Drawing on findings that carceral
contact can drive perceptions of carceral state illegitimacy and contact
group solidarity, I suggest that it may too generate perceptions of
carceral targets as more deserving. Moreover, impacted groups learn
these political messages by interacting with policy design. They may in
turn have different reference points for carceral policy, and the
conditions and consequences of imprisonment. In turn, I expect that
carceral contact will translate to more generous perceptions of
deservingness of, and consistent support for policy that helps carceral
citizens.

I test these propositions using original 2024 survey data capturing
Americans' experiences with the carceral system, their perceptions of a
range of policy targets, and their attitudes towards prison-related
policy reforms. The data allow for a unique empirical examination of how
deservingness judgements may vary across direct and indirect experiences
with incarceration and victimization -- and how these judgements
structure policy preferences.

In doing so, this paper makes three contributions. First, it extends
theories of social constructions and deservingness -- the latter of
which was largely developed in the context of welfare and redistributive
policy - into the domain of criminal legal policy. Second, it sheds new
light on how policy or institutional experience and proximity modifies
deservingness perceptions and their correspondence to substantive policy
views. Last, it responds to recent interventions that experiences of
victimization constitute a critical but understudied component of this
feedback loop (Levine and Russell, n.d.) and deepens our understanding
of how victimization as a kind of proximal contact of its own can alter
the public's orientation toward punishment, policy, and state
legitimacy. Understanding how impacted groups differ in their approach
to policy -- and whether the preferences of most victims and
perpetrators are as far apart as dominant narratives would suggest -- is
important not only for scholars but also for advocates and policymakers
pursuing policy reform in this domain.

The remainder of this paper proceeds as follows. In section 2, I
elaborate first on key findings from social constructions and
deservingness, before discussing policy feedback literature and its
implications -- particularly, the literature's its conceptualization of
policy targets as active agents in meaning-making processes and
empirical findings in the criminal legal domain -- for my expectations
about criminal legal policy. I develop theoretical framework, outlining
how carceral contact may shape perceptions of deservingness and how
those perceptions may, in turn, influence policy views. Section 3 then
presents the data and analytic strategy, drawing on a newly fielded
national survey to test these relationships. I conclude in section 5
with a brief discussion of implications for theories of deservingness,
policy feedback, the politics of punishment.

\subsection{Deservingness and Attitudes Towards Public
Policy}\label{deservingness-and-attitudes-towards-public-policy}

How does contact with the criminal legal system affect perceptions of
its targets? And what do these deservingness perceptions mean for policy
views? Scholarship of the social constructions of target populations and
deservingness respectively examine how groups become characterized as
deserving or undeserving of government assistance, and they demonstrate
that people's policy preferences often hinge on such characterizations
(A. Schneider and Ingram 1993b; Smith and Kreitzer 2024; W. van Oorschot
2006a).

Schneider and Ingram's (1993) theory of social constructions locates
perceptions of target deservingness in a dialectic with policy design.
That is, public perceptions of whether policy targets are deserving or
undeserving are informed by social constructions, or ``value-laden
components, including stereotypes, dominant ideologies, and
assumptions'' about groups (Smith and Kreitzer 2024, p.~640).
Policymakers are attentive to these perceptions, as they expect their
constituencies to electorally reward them for producing policy that
helps deserving groups like veterans, and refraining from offering help
to or punishing the least deserving -- particularly ``deviant'' groups
like carceral citizens\footnote{Schneider and Ingram's theory of social
  construction goes on to note that policymakers respond not only to
  their constituencies' perceptions of target deservingness, but also to
  the political power of target groups, or their ability to ability to
  access, reward, and/or retaliate against policymakers. These
  interacting dimensions (power and deservingness) produce four
  categories of groups in society, to whom policy makers respond with
  distinct policy. Policymakers are incentivized to help groups who are
  ``advantaged'' -- both powerful and perceived as deserving -- and
  ``contender'' groups, or those who are powerful but constructed as
  undeserving. In contrast, policymakers offer burdensome policy to
  ``dependents'' -- those who are seen as deserving but who are not
  politically powerful, and punish ``deviants,'' who are seen as neither
  deserving nor powerful. Political representatives' policy design is
  also driven by ``anticipatory'' feedback: policymakers respond to both
  their perception of groups' political power and social constructions
  in terms of electoral consequences -- for example, by providing
  ``hidden'' benefits to contenders, who are constructed as undeserving
  but powerful, policymakers can avoid political blowback for helping an
  ``undeserving'' group while strengthen their own political position.
  (Schneider and Ingram 2019, Ellis and Faricy 2020).}. Thus, policy
re-enforces perceptions of deservingness, by ``telegraph{[}ing{]} to the
public how target groups should be treated'' (Smith and Kreitzer 2024,
p.~640). While, as a consequence, particularly negative social
constructions like those of carceral citizens may become ``sticky'' or
more entrenched, social constructions are also contingent, for example,
on local, state, and political and cultural contexts and also on the
political identity of and ideology held by the `perceiver' (Smith and
Kreitzer 2018, p.~768). Thus, the relevant social construction driving
perceptions of deservingness may different across audiences and policy
contexts.

While social construction scholars helps illustrate external mechanisms
that may drive broad constructions of carceral citizen's deservingness,
others have paid more attention to specific mechanisms shaping how
people judge whether policy targets are deserving. This line of
deservingness scholarship explains that people form quick judgements
about policy targets' deservingness by evaluating constructions along
five dimensions in an evolutionarily-grounded impulse to help
``reciprocators'' and avoid ``free-riders'' (Petersen et al. 2012,
2012). These ``CARIN'' criteria primarily attend to cues about whether
recipients cannot help their circumstances, or whether they are seeking
unearned benefits. They refer to: \emph{control,} which in its original
conception was closely related to \emph{need}, referring to whether
someone is perceived as lazy or unlucky -- a consideration Oorschot
(2006) argues can override all others. More recently, perception of a
group or individual's \emph{need} for assistance has considered the
severity of their neediness and \emph{control,} or the degree to which
they are responsible (SoRelle and Laws 2024). Perceptions of the group's
\emph{attitude} refer to whether the beneficiary is grateful and
likeable, and closely related perceptions of the group's
\emph{reciprocity} connote their history or apparent willingness to
contribute to society. Last, the \emph{identity} characteristic holds
that people will evaluate more generously the deservingness of others
whose identity is aligned with their own (Oorschot 2000, 2006). While
this framework originated to explain views of welfare policies, scholars
have applied it to explain policy views in other areas of social policy,
including health care (Gollust and Lynch 2011).

\subsection{From Social Constructions of Deservingness to Policy
Feedback}\label{from-social-constructions-of-deservingness-to-policy-feedback}

My theoretical starting point is that perceptions of what groups like
carceral citizens deserve are often shaped by social constructions about
them (Schneider and Ingram 1993) . Like stereotypes, these constructions
are produced by policy and discourse -- both political, cultural, and
popular -- and they are often taken up by individuals in a heuristic
process to form quick judgements about what a target group deserves.
Thus, predominant racialized constructions of incarcerated people --
``thugs,'' ``felons,'' or ``inmates,'' -- that invoke ingratitude,
self-orientation, and blameworthy-ness for their condition of
confinement are likely interpolated as undeserving.\footnote{In the
  criminal legal domain, these representations are often shaped by
  racialized media portrayals and political discourse that emphasize
  danger and individual blame (Lopez 2015; Vasiljevic and Viki 2014).}

Desesrvingness literature may suggest one direct pathway through which
contact might interrupt this process. Simply put, someone who is in and
knows people in prison has access to information about who an
incarcerated personal is outside of their ``criminal'' identity in a way
that someone who has only seen ``prisoners'' on TV may not. Carceral
state contact should, by definition, change what information individuals
have about its targets. This alone could translate to more generous
interpretations of deservingness (W. van Oorschot 2000b; W. van Oorschot
2006b; Feather and McKee 2009). \footnote{For example, contact may lead
  individuals to see incarcerated people as facing substantial hardships
  (\emph{need}), with less control over the conditions that led to
  incarceration (\emph{control}), and as capable of gratitude, growth,
  or giving back (\emph{reciprocity}, \emph{attitude}). In contrast to
  dominant constructions that depict incarcerated people as undeserving,
  contact may ``fill in'' these criteria with lived or vicarious
  knowledge. On this dimension, then, contact may generate warmer
  deservingness perceptions by replacing stereotypes with personalized,
  alternative narratives about who carceral citizens are and what their
  conditions and the consequences of confinement.} Following Schneider
and Ingram (1993), these deservingness perceptions then influence how
individuals evaluate related criminal legal policies.

Yet it is important to note that this logic implicitly assumes that the
processes by which deservingness is interpreted and shapes policy views
operate the same way in the criminal legal domain as they do in the
welfare state, where much of deservingness literature originated. Recent
work complicates that assumption. Schneider and Ingram (2019) note that
even ``deviant'' groups -- long presumed to be negatively constructed --
can become beneficiaries of policy help, especially when costs,
visibility, or broader group associations shift. SoRelle and Laws (2023)
observe that perceptions of individual deservingness may not linearly
translate into support for policy overall. Together, these insights
suggest the need not only to examine whether contact reshapes
perceptions of deservingness, but also to assess how --- and whether ---
those perceptions relate to support for state intervention in the
carceral domain. Yet what remains less fully theorized is whether, and
how, the meaning and influence of these social constructions differ for
individuals who have direct experience with the relevant policy. Might
experiences with policy change how these constructions and
characteristics are interpreted, or the extent to which they are
internalized and interpolated into policy views? Lessons from policy
feedback literature suggest they might.

Across domains, including the carceral state, policy feedback literature
shows that policies can shape attitudes towards the state and
institutions, one's sense of ``civic self,'' and political behavior as
part of a process through which policies, once established, have the
capacity to shape and reshape policy landscapes (Campbell 2012; Skocpol
1992). These ``feedbacks'' operate through two core mechanisms: resource
effects, which concern the distribution of material benefits and costs,
which can enact barriers or incentives to participation, and
interpretive effects, which concern the ability of policy design to
convey messages to its targets about their civic standing, and what they
can expect and may ask of government as a whole (Campbell 2011; Mettler
and Soss 2004; Mallory SoRelle and Serena Laws 2022).

This interpretive element offers an extension of social construction
theory, in that it suggests that the messages a policy target infers
from interacting with policy are unique, and should not be inferred to
mirror the messages policy conveys about to broader audiences its
targets. Moreover, this framework positions policy targets as ``active
agents'' in the construction of meaning who internalize and respond to
policy messages in different ways -- by rejecting a negative
construction or its policy implication and engaging in collective
action, or conversely by differentiating onesself and disaggregating
one's overall policy views from their perceptions of beneficiary
deservingness (Soss 1999a; Goss 2012; Thurston 2018; SoRelle 2020;
Lacombe 2022).\footnote{Soss (1999) explains a tendancy of recipients of
  means-tested welfare (AFDC) to maintain negative constructions about
  beneficiaries of the same policy (unlike recipients of the non-means
  tested disability benefits) as a product of political learning while
  on the AFDC: benefits are burdensome to access, involve stringent
  behavioral monitoring, and involve more caseworker discretion than
  standardized rules, conveying to AFDC beneficiaries that government
  views them as responsible for their need and untrustworthy. This leads
  beneficiaries to not only internalize a more ``undeserving''
  perception of their own beneficiary group, but also splinters any
  sense of group solidarity and disaggregates policy views from
  perceptions of beneficiary deservingness.}

In the criminal legal domain, scholarly debate around the conditions
under which carceral contact is politically mobilizing or demobilizing
foregrounds this variation (cf.~White 2022). Many carceral citizens,
their networks, and neighborhoods internalize the clear ``undeserving''
messages that prison policy sends (Lee, Porter, and Comfort 2014; Weaver
and Lerman 2010b). In doing so, feedback scholars also dilineate the
experiences and outcomes of people with not only direct, but also
indirect or ``proximate'' contact, such as families of targets and
neighborhood effects. For example, (Burch 2014) finds that high
concentrations of incarceration and community supervision reduce
political participation, creating ``important spillover effects that
suppress participation not only of the supervised individual but also of
those living around him or her'' (197). But others find that both direct
and proximal contact can mobilize (A. Anoll and Israel-Trummel 2019a; H.
L. Walker 2020b; Ewald 2024). While these studies do not focus
specifically on the link between contact and deservingness, they are
nonetheless theoretically instructive about the role of direct and
proximate contact on political behavior.

While many carceral citizens and their networks internalize the clear
``undeserving'' messages conveyed by criminal legal policy, research
increasingly shows that such contact can also shape political attitudes
towards the state -- fomenting perceptions of institutional illegitimacy
(A. Anoll and Israel-Trummel 2019a; A. P. Anoll, Epp, and Israel-Trummel
2022a; H. L. Walker 2020a; Lerman and Weaver 2014a). As such, I propose
that interactions with the carceral system should also shape attitudes
towards \emph{targets of the state} -- and their legitimacy or
deservingness in relation to it. While some work finds that targets of
policy can internalize negative or undeserving constructions (Soss
1999b), much of the work on criminal legal contact stresses that
demobilization is a product of resource effects or interpretive effects
\emph{distinct} from an ``undeserving'' self-conception (Lerman and
Weaver 2014, White 2019). Instead, scholars emphasize that direct
contact with the criminal legal system can heighten in-group solidarity
and mobilize in spite of resource and interpretive feedbacks that should
be demobilizing. (Walker 2020, Owens 2014). This is specifically the
case when policy targets reject policy ``teachings'' about their lower
civic status as part of a sense of systemic injustice -- suggesting a
reappraisal of both the system and those criminalized by it (Walker
2020). Among family of the incarcerated, Anoll (2022) suggests that
higher rates of voting in states specifically with most stringent
carceral disenfranchisement policies suggests ``surrogate''
mobilization, or a motivation to act on behalf a ``deserving''
incarcerated citizen, is at work. Thus, mobilization in response to both
direct and proximal contact involve meaning-making processes that wed
social solidarity with dampened perceptions of institutional legitimacy
and trust.

But interaction with the carceral system may facilitate not just
political, but also policy learning. Targets and their networks come to
learn policy messages by experiencing policy burdens. Lee et al.~(2014),
for example, find that demobilizing 'spillover effects'' are trace to
family and loved ones of the incarcerated experiences with and reaction
to policies whose costs they incur, like those of visitation, expensive
phone calls, and paying off fines and fees. That the experiences
producing political responses are grounded in personal experience with
policy suggest that contact may reorient how individuals think about
policy, as it changes what they think (for example, what consequences to
policy they imagine) when they think about policy.

Feedback scholars acknowledge endogeneity challenges inherent to
questions of carceral state contact in a context of ``cumulative
burdens'' or where the same individuals and communities often experience
both direct and proximate carceral contact (Michener 2018; Weaver and
Lerman 2010c; Lerman and Weaver 2014b). These experiences in isolation
may have important differences. While some note psychological
motivations the incarcerated may have to hold deserving perceptions of
themselves and others (Maltby and Kreitzer 2023),\footnote{Maltby and
  Kreitzer (2023) also take up the question of carceral contact's effect
  on target deservingness, several limitations in the design raise
  ques-tions about whether their findings reflect a measurement
  artifact. First, the study conflates arrest with incarceration,
  despite these experiences carrying distinct resource effects, stigma,
  and lessons vis-a-vis one's civic standing and government
  responsiveness (cf.~Soss 2005). Further still, arrest may allow for
  self-distancing and meritocratic narratives, whereas incar-ceration
  deeply implicates individuals in systems that mark groups as
  ``undeserving'' (cf.(Newman, Shah, and Lauterbach 2018))5. Second, the
  study aggregates deservingness per-ceptions of ``criminal legal
  targets'' that includes groups like ``opioid users,'' ``welfare
  cheats,'' and other groups less apt to the relationships of interest:
  how policy experience shapes per-ceptions of policy target
  deservingness. Third, their design does not test what these
  deserv-ingness mean for policy preferences, which given its novel
  application of deservingness to the carceral contexts, leaves open the
  question of whether these perceptions mean the same thing in the
  criminal legal domain. Further still, the study does not distinguish
  between dif-ferent types of contact, such as incarceration versus
  victimization. These limitations reflect a broader pattern in the
  literature, where the effects of carceral contact --- especially forms
  like victimization --- remain theoretically and empirically
  underdeveloped.} others stress that they may be far more likely to
internalize undeserving messages, from which the proximally impacted may
be more removed (A. Anoll and Israel-Trummel 2019b; A. P. Anoll, Epp,
and Israel-Trummel 2022b; H. L. Walker and García-Castañon 2017, 2017;
M. L. Walker 2016; H. L. Walker 2014). Experiences of victimization
amplify this tension: they often occur alongside policing and
incarceration in the same communities, yet are rarely conceptualized as
a form of policy contact in their own right.\footnote{One notable
  exception is Levine and Russell (2023), who show how the design and
  financing of victim compensation law constructs victims as morally
  deserving and politically worthy, helping expand state responsibility.
  Yet even here, the analysis centers on state building and symbolic
  policy meaning---rather than how victimization itself may function as
  a form of lived experience that may alter how individuals perceive the
  state, its targets, and what policies they deserve.}

Although victims of crime are rarely theorized as policy feedback
subjects, the concept of victimization is implicitly central to the
literature. Policy feedback scholars emphasize how individuals interpret
state actions---especially punitive or discriminatory ones---as signals
of exclusion, abandonment, or devaluation (Soss 1999c; Lerman and Weaver
2014c). These ``interpretive effects'' can leave people feeling
targeted, disrespected, or undeserving in the eyes of the state. In this
context, experiences of secondary victimization---such as being
disbelieved, ignored, or emotionally constrained by the criminal justice
process (Lensa et al., n.d.) --- can function much like the demobilizing
interpretive effects faced by other negatively constructed policy
targets. Yet unlike welfare recipients or formerly incarcerated people,
crime victims are more often assumed to support the very policies and
institutions that may reproduce their harm. Recognizing victimization as
a form of state contact---and a potential site of interpretive
rupture---opens new terrain for understanding how criminal legal
experience shapes not only institutional trust and political
participation, but also deservingness perceptions and carceral policy
attitudes. This paper takes up that challenge by analyzing victimization
alongside other forms of contact to assess how they each influence
perceptions of who deserves help or harm---and what role the state
should play.

\subsection{Hypotheses}\label{hypotheses}

How does contact shape perceptions of what carceral citizens deserve?
While views on criminal legal policy and how people conceive of carceral
citizens are complicated questions, I anticipate that these experiences
(incarceration, neigborhhood or social connections to the incarcerated,
and victimization), depress the authority of carceral policy design to
convey meaning about its targets. Perceptions of systemic injustice and
trends of surrogate participation suggest contact may generate more
nuanced perceptions of deservingness in its own right, as carceral
citizens are conceived of as victims of injustice and or deserving of
representation. While individuals within these groups may still respond
to policy messages and think about targets in different ways, I expect
these groups to view carceral citizens as more deserving than those
without impact. Because people with proximate impact may be less likely
to internalize undeserving messages, they may have warmer perceptions
than the directly impacted. Further, I suggest that experience with the
criminal legal system -- including victimization -- leads people to
stronger and more consistent support for offering policy help to help
incarcerated people, and to more nuanced ways of thinking about the
relationship between what a carceral target `deserves' and what policy
is appropriate.

Respondents with no direct or proximate experience with incarceration
may rely more heavily on dominant public narratives that portray
incarcerated individuals as violent, morally deficient, or socially
deviant. Social construction theory suggests that, in the absence of
countervailing personal experiences, people turn to culturally available
representations to evaluate policy targets (Schneider and Ingram 1993;
Smith and Kreitzer 2024). In the criminal legal domain, these
representations are often shaped by racialized media portrayals and
political discourse that emphasize danger and individual blame
(Ghandnoosh 2014). Without lived or relational knowledge to ``fill in''
CARIN criteria such as control, reciprocity, or need with humanized
nuance, these respondents are more likely to default to punitive or
stigmatized constructions of incarcerated people. Even those who know
only victims of crime may be exposed to narratives emphasizing offender
culpability and just deserts---without the complexity that comes from
understanding incarcerated people as multifaceted individuals.
Therefore, I expect that:

\textbf{\emph{H1}} \emph{Direct and proximate experiences with
incarceration and experiencing victimization will lead to more generous
deservingness perceptions of incarcerated people.}

\textbf{\emph{H1-A}} \emph{Individuals with proximate impact will report
more deservingness perceptions of incarcerated people. than those who
are directly impacted.}

While victimization might intuitively drive negative perceptions of
carceral citizens, I suggest the relationship is more complex.

Researchers provide evidence that more abstracted, dehumanized
stereotypes about carceral citizens (Vasiljevic and Viki 2014) generate
harsher assessments of culpability (Levinson 2007; Levinson, Cai, and
Young 2010; Donovan 2007; Israel-Trummel and Streeter 2022) and
sentences (Rehavi and Starr 2014) suggesting that information
\emph{besides} conviction and crime about carceral citizens should have
a countervailing humanizing effect (Manza, Uggen, and Brooks 2006).

However, there are clear reasons why victimization may generate
\emph{undeserving} perceptions: victimization is highly a salient
experience that for many drives negative affect toward those who commit
crimes (Ditton et al. 1999) and provides a clear justification to focus
on criminal culpability when considering incarcerated people overall
(Culhane, Hosch, and Weaver 2004). However, this logic assumes clear
victim-offender boundaries and risks ignoring the broader social context
in which violence occurs. The overlap between victims and otherwise
impacted groups is significant (Jennings, Piquero, and Reingle 2012).
Victims of violent crime are disproportionately part of communities most
heavily policed and incarcerated and most know their offender (Bureau of
Justice 2024) -- most individuals who experience incarceration have also
been victimized, as well.\footnote{An extensive literature shows that
  people who are incarcerated have disproportionately been victims of
  crime -- one review article cited 31 of 37 papers support this overlap
  (Jennings et al 2012). While 2\% of the general US population report
  being victims of violent crime, up to 45\% of carceral citizens have
  experienced pre-incarceration physical abuse, and 8.5 to 39.2\% of
  specifically sexual abuse (Azimi et al 2019, Carlson and Shafer 2010,
  Messina et al 2007, Wolff and Shi 2012, Yoder et al.~2017). Still more
  experience violence while incarcerated (Wolff et al 2009).} Thus,
while for many victims the proximity and salience of a `criminal' may
bolster negative social constructions and deservingness perceptions, it
might also translate to more nuanced perceptions of culpability and a
closer alignment of identity that moderates negative affect.

Moreover, victims face restrictive, ``sticky'' social constructions --
what Lens et al (2014) note generate negative feedbacks. While
navigating narrow expectations of ``appropriate'' emotional responses in
the criminal justice process -- non-emotional victims risk disbelief
while overly emotional victims face judgment -- victims experience
``secondary victimizations'' that foment institutional distrust. In this
way, victims may come to share with incarcerated individuals not only
carceral exposure, but also skepticism toward the system itself and its
moral judgments about who is or isn't deserving (Walker 2020; Anoll \&
Israel-Trummel 2019). For some, this may produce more nuanced
assessments of incarcerated people as victims of systemic injustice
rather than mere perpetrators. For others, especially those who strongly
identify with retributive justice or lack personal connection to
offenders, deservingness perceptions may remain harsher. However, it is
worth noting that victimization generally does not increase punitiveness
-- and that instead perceptions of crime are more important in driving
punitive sentiment than actual crime - suggesting victimization may both
lead to more moderated perceptions of carceral citizens and perceptions
of what they deserve (Hale 1996; Kleck and Jackson 2017a, 2017b). I
therefore expect that:

\emph{\textbf{H1-B} Victims of violent crime may vary more than other
impacted groups in their perceptions of incarcerated people's
deservingness, but on average, will still express more generous views
than individuals with no direct or proximate contact with the carceral
system.}

Social construction theory suggests that perceptions of target
deservingness translate directly to policy preferences: groups
constructed as undeserving receive punitive policy while deserving
groups receive beneficial policy (Schneider and Ingram 1993). Applied to
the carceral domain, this means that warmer perceptions of incarcerated
people's deservingness predict greater support for policies that help
rather than punish them. Thus, deservingness acts as a mediating bridge
between group perception and carceral policy support. I therefore
hypothesize that:

\emph{\textbf{H2} More generous perceptions of carceral citizens'
deservingness will predict greater and more consistent support for
assistance-oriented criminal legal policies and reduced support for
punitive criminal legal policies.}

However, evidence from SoRelle and Laws (2023) and others suggests the
relationship between deservingness perceptions and policy support may
not be straightforward in all contexts. The criminal legal domain may
have unique dynamics where perceptions of groups and support for
policies directed at them operate through different mechanisms.

\subsection{Data and Methods}\label{data-and-methods}

I explore whether policy contact shapes people's attitudes about
deservingness, and how those deservingness perceptions mediate policy
preferences. I employ original survey data using a national sample of
2,716 US adults collected in September 2024 through the platform
Forthwright. The survey adapts the approach of Kreitzer and Smith's
(2018) empirical mapping of power and deservingness constructions of 87
target populations.\footnote{The survey expands on Kreiter and Smith
  (2018)'s approach by recruiting from a Bovitz proprietary panel
  (although the survey was hosted and deployed from Forthright) and
  reducing the number of groups individuals access to reduce issues of
  respondent fatigue. 2,716 eligible participants were recruited to
  complete the 30-minute survey on the platform Forthright, and were
  paid 10 dollars. The sample is just under 51\% women and 49\% men, and
  leans more Democratic than the national average (39\% versus 31\%) but
  is equally Republican (25\%). The underrepresentation of Republicans
  may have a minimizing effect. I apply exclusion restrictions for low
  response quality, failing to pass all three attention checks, and
  extreme outlier responses, and also create two response quality
  variables for robustness checks, resulting in a final sample of
  {[}2,286{]}. Sample details are in Table 1 of the appendix.}

\textbf{Measuring Deservingness Perceptions, Contact, and Policy Views.}

In this study, I investigate the correlation between criminal legal
policy experience and perceptions of criminal legal target deservingness
as well as views on criminal legal policy, and the extent to which
deservingness perceptions mediate policy views. As such, my primary
variables of interest concern deservingness ratings for target groups,
``contact'' with the criminal legal system, and policy views.

Perceptions of target deservingness are operationalized using 0 to 100
sliding scale ratings of ``incarcerated people/prisoners.'' After
reading a short explanation of what deserving usually connotes (Figure
), respondents rated the deservingness of a total of 65
groups.\footnote{Survey participants were randomly assorded into three
  sections (A, B, and C). All respondents saw the same 45 groups. Each
  section then reviewed an additional (20) unique groups.}

\begin{tcolorbox}
\textbf{Survey Item Prompt}

Some people, groups, and organizations are viewed as contributing to the general welfare of society and worthy, and thus are deserving of sympathy, pity, or help. Typically, we describe members of these groups as good, smart, hardworking, loyal, disciplined, generous, caring of others, respectful, and creative.

Meanwhile, there are many other groups that are viewed as a burden to the general welfare of society, and are believed to be undeserving of sympathy, pity, or help. Typically, we describe members of these groups as greedy, disrespectful, disloyal, immoral, disgusting, dangerous, lazy, and expect others to care for them.

Based on what you know about these groups, how deserving or undeserving would you say each of these groups are, generally speaking? Here, 0 means most people in that group are completely undeserving. 100 means most people in that group are very deserving.
\end{tcolorbox}

I argue a respondent's experience with the criminal legal system affects
perceptions of carceral citizens' deservingness and policy views. To
measure policy contact and proximate contact, I ask respondents whether
they or someone they know well have been ``incarcerated in jail or
prison,'' and/or whether they and/or someone they now have ``been a
victim of violent crime.'' I construct three binary indicators for the
three contact types of interest, which represent respondents who have
been incarcerated (1), know someone incarcerated (1), or have been
victimized (1), or not (0). I do not isolate for individuals who have
only had one form of policy experience -- much of the incarcerated
sample overlaps with those who know incarcerated individuals, as would
be expected\footnote{I also expect that respondents under report both
  victimization and incarceration because they are stigmatized
  experience, introducing random noise (Skogan 1986).}. Additionally, I
construct a single ``impact'' dummy to capture all three impacted groups
to simplify analysis for my first overarching hypothesis.

Finally, to explore how contact shapes views on policy views and
moderates deservingness' mediation of policy views, I ask respondents
their Likert-scale support/opposition to six items on criminal legal
policy. Two items (support for the death penalty and for life without
parole) are combined in a `prison penalty' support index, and the
remainder which involve policy help (minimum wage pay requirements for
incarcerated workers, voting rights restoration, etc) are combined in a
`prison help' support index.

\paragraph{Other Factors Influencing Policy
Views}\label{other-factors-influencing-policy-views}

I include standard additional controls and two unique controls to attend
alternative explanations of perceptions of deservingness, policy views,
and their relationship. First, Smith and Kreitzer (2024) show that while
there are different levels of across and within partisan consensus
around target group deservingness, partisanship strongly influences
target social constructions. To control for party affiliation,
respondents indicate whether they are Republican, Democrat, Independent,
or `other party.' I code for Republicans and Democrats---leaving
independents and ``others'' as the comparison groups. Respondents also
indicate whether they are very liberal, somewhat liberal, moderate,
somewhat conservative, or very conservative. I code this as a five-point
ordinal variable, with higher values indicating more conservative
ideological orientation\footnote{Research consistently shows that
  political conservativism is associated with stronger support for
  punitive policy, and also stronger and more frequent support for
  guilty verdicts and harsher sentencing recommendations as well as
  harsher prison conditions (Green 2012, Pyo 2024, Hansen and Navarro
  2024). These may refllect beliefs in personal culpability, expressive
  retributive justifications for punishment, and value commitments to
  law and order that are often fundamental to conservative ideology
  (Burton et al 2020, Mancici et al 2021, Wilson et al 2015).}.

Some carceral state scholarship suggests that individual views about
carceral citizens and policy are determined by punitiveness towards
offenders -- whether conceptualized as beliefs about meritocracy and the
rule of law, or expressive responses to perceived wrongdoing or harm
(Sniderman and Piazza 1993, Chouhy et al 2022, Miller and Alexander
2016). While some operationalize this via support for the death penalty
(Enns 2016), others use measures that reflect belief in the purpose of
punishment. Here, I operationalise this concept with a control for
beliefs about the purpose of punishment using an interval `penal point'
variable, with respondents' views on whether the purpose of
incarceration as to punish primarily, rehabilitate primarily, or
somewhere in between, rescaled to a 0-1 metric from a 7 point scale.

A broader ``punitive'' orientation may influence perceptions of what
incarcerated people deserve. By this I mean that policy targets like
incarcerated people may, for some, be implicitly constructed as being in
need of help or intervention, socially transgressive, or unfamiliar or
distant from mainsteam society. As such, deserving perceptions of
incarcerated people may reflect broader attitudes towards these groups
overall. Similarly, respondents may give these groups ratings that
reflect ``deserving of some inferred benefit,'' where others groups,
like ``Athiests,'' ratings may reflect a more generalized perception. To
account for this general tendency, I construct a control variable that
captures respondents' deservingness ratings of groups that are framed in
public discourse as needing help---such as welfare recipients, Medicaid
and Medicare/SSN recipients, the unemployed, poor families, homeless
individuals, and asylum seekers or refugees. Specifically, I average
their raw 0 to 100 deservingness ratings for the select subset groups.

To ensure that I do not control for a variable that is conceptually or
empirically too close to the deservingness of incarcerated people when
it is the dependent variable, I exclude any deservingness ratings for
groups that are substantively overlapping or highly correlated.
Including those could absorb variance in the outcome or introduce
post-treatment bias by controlling for a related evaluative construct. I
also avoid using global composite or average deservingness scores across
all groups -- for example, a respondent's mean deservingnesss rating
across all 65 groups they evaluated -- because that composite would
include the rating of incarcerated people itself, introducing potential
collinearity and making it difficult to isolate the unique effect of
contact. This modeling strategy attempts to preserve interpretability
and ensuring that the effect of policy contact is estimated with respect
to unstandardized, target-specific deservingness scores.\footnote{While
  constructing composite or control variables involving other
  deservingness targets, I assess whether their inclusion introduces
  collinearity or conceptual overlap with the dependent variable. In
  particular, I examine correlations between group-level deservingness
  ratings to ensure that any included groups are not too closely tied to
  perceptions of incarcerated people. If preliminary checks reveal that
  including a particular group (e.g., ``criminals'' or ``formerly
  incarcerated individuals'') would obscure key variance or conflate
  target constructs, I exclude it from the model and treat it as
  conceptually non-distinct for the purposes of this analysis.} This
control variable helps isolate whether the relationship between criminal
legal contact and deservingness perceptions of incarcerated people
reflects a target-specific shift in perception, or a broader orientation
toward helping marginalized groups. In doing so, it sharpens the test of
H1 by clarifying whether contact affects views of \emph{carceral
citizens in particular}, rather than simply increasing general
generosity toward a wider net of populations.

Similarly, some respondents may be less supportive of policies that help
such groups. To account for broader ideological dispositions toward
state assistance, I construct a control variable capturing general
support for redistributive or assistance-oriented policies. This policy
disposition index is based on a set of standardized survey items,
including support for Medicaid expansion, unemployment assistance,
increasing the minimum wage, government responsibility for helping the
poor, and agreement with statements about the importance of accepting
refugees. I helps assess whether the relationship between criminal legal
contact, deservingness perceptions, and policy views reflects
domain-specific evaluations of carceral targets---or simply broader
support for government assistance. In doing so, it sharpens the test of
H2 by clarifying whether deservingness meaningfully predicts policy
preferences \emph{beyond} respondents' general redistributive
orientation.

I will also control for respondent race and racial attitudes.
Respondents are asked whether they are White, Black or African American,
Native American/Alaska Native, Asian/Asian American, Native
Hawaiian/Other Pacific Islander, Middle Eastern/North African,
Hispanic/Latino/a/e, or ``my preferred response is not listed.'' I will
use a series of indicator variables (non-Hispanic white, non-Hispanic
Black, Hispanic/Latino, and ``other'' race). I control for variation in
respondents' attitudes using the four-item FIRE battery that captures
both cognitive and affective components of racial attitudes: fear,
acknowledgement of institutional racism, and racial empathy (DeSante and
Smith 2020). I use FIRE in place of the traditional Kinder scale because
the latter captures only narrow ideological dimensions that may not
capture a distinct role of emotional responses like fear and empathy in
racialized policy domains.

Finally, I also control gender, level of education, and location. I
control for gender with a binary variable where respondents who
identified as genderqueer or `other' (of which there were relatively
few) will be randomly assigned to woman (0) and men (1).

\subsection{Analysis}\label{analysis}

I use the above described data to test my four hypotheses with a series
of regression models. To evaluate the effect of criminal legal contact
on desesrvingness perceptions, I perform several additional regressions
while controlling for attitudinal conditions and demographic and
political factors. This sequence allows me to test robustness and
interactive effects, while building toward full specifications.

It is useful to begin by assessing broad patterns in perceptions of
incarcerated people and support for criminal legal policies across
impact groups. I anticipate that people who have experienced criminal
legal contact will be more generous in their deservingness ratings of
incarcerated people and more consistent in their support across policies
that help them. To preview these patterns, I (will) first present the
mean deservingness scoress (0-100 scale) of incarcerated people across
four impact groups: those who have been incarcerated, those who know
someone incarcerated, those who have experienced violent victimization,
and those with no direct or proximate contact. I also include a fifth
group---those who know victims---to examine the effects of victimization
exposure alone.

To explore policy opinion variation, I will next present a table showing
mean levels of support for prison-assistance policies and other
decarceral reforms across the same impact groups. Column A focuses on
support for policies that directly assist incarcerated individuals
(e.g., education, family support), while Column B reports strictly
punitive policy (life without parole and the death penalty). These
descriptive patterns offer a first look at the extent to which contact
correlates with policy orientations.

\paragraph{\texorpdfstring{\emph{Modeling
Approach}}{Modeling Approach}}\label{modeling-approach}

To formally evaluate my hypotheses, I fit a series of Ordinary Least
Squares (OLS) models examining the relationship between carceral
contact, deservingness perceptions, and support for criminal legal
policies. Unless otherwise stated, all models include relevant
demographic and attitudinal covariates, denoted in vector form as
\(X_{i}^{\prime} \cdot \texttt{controls}\),. In all models, \(i\)
indexes individual respondents, and \(\texttt{impact}_{i}\) captures
whether a respondent has been directly or proximally impacted (included
via victimization). In models where all key variables are measured
within the same experimental block, standard errors are not clustered.
Where appropriate (e.g., models pooled across blocks or with repeated
measures), I cluster standard errors by respondent. I interpret
coefficients of interest using two-tailed hypothesis tests at the
\(\alpha = 0.05\) level. I report 95\% confidence intervals and standard
errors throughout.

\subparagraph{Deservingness Models}\label{deservingness-models}

My first main hypothesis (\textbf{H1)} predicts that individuals with
direct or proximate experiences with the criminal legal system (e.g.,
incarceration, knowing someone incarcerated, or victimization) will
perceive incarcerated people as more deserving. I test this by
regressing each respondent's 0-100 deservingness scores for incarcerated
people on a binary indicator for carceral impact status, controlling for
demographic and attitudinal factors. Specifically, I estimate:

\subparagraph{Model 1: Any Carceral Impact on
Deservingness}\label{model-1-any-carceral-impact-on-deservingness}

\[
{\texttt{prisonerdeserves}}_{i} = \beta_0 + \beta_1 \cdot {\texttt{impact}}_{i} + \beta_2 \cdot {\texttt{desor}}_{i} + X_i^{\prime} \cdot {\texttt{controls}} + u_i
\]

where \(prisonerdeserves_i\) is respondent \(i\)'s deservingness score
for incarcerated people, \(impact_i\) is a binary indicator for whether
the respondent has been impacted in any of the stated ways, and \(X_i'\)
includes demographic data. The variable \(desor_i\) captures each
respondent's general orientation toward help-seeking populations. The
key coefficient of interest is \(\beta_1\), which estimates the average
difference in deservingness perception between those with and without
carceral contact, holding other covariates constant. A positive and
statistically significant \(\beta_1\) would support H1.

While not central to my theoretical framework, I also explore whether
the effect of contact on deservingness perceptions varies by political
ideology. This model is reported in Appendix Table A2.

To I distinguish among types of contact (H1, H1A,H1B), I estimate
separate models in which the impact variable is disaggregated into
binary indicators for each type (e.g., been incarcerated, knows someone
incarcerated, victim of violent crime). This allows me to test which
forms of contact are associated with more generous perceptions, and
whether the effects differ in magnitude or direction.

\subparagraph{Model 2: Incarcerated, Know Incarcerated, Victimized on
Deservingness}\label{model-2-incarcerated-know-incarcerated-victimized-on-deservingness}

\[
\texttt{prisonerdeserves}_{i} = \beta_0 + \beta_1 \cdot \texttt{incarc}_{i} + \beta_2 \cdot \texttt{knowincarc}_{i} + \beta_3 \cdot \text{vict}_{i} + \beta_4 \cdot \texttt{des\_or}_{i} + \beta_5 \cdot \texttt{controls}_{i} + u_{i}
\]

\subparagraph{Policy Models}\label{policy-models}

My second core hypothesis (\textbf{H2)} is more generous perceptions of
carceral citizens' deservingness will predict greater and more
consistent support for assistance-oriented criminal legal policies and
reduced support for punitive criminal legal policies. To test this, I
estimate the following outcome models for each policy index:

\subparagraph{Models 3 and 4: Deservingness on Carceral
Policy}\label{models-3-and-4-deservingness-on-carceral-policy}

\[
\texttt{prisonhelp}_{i} = \delta_0 + \delta_1 \cdot \texttt{impact}_{i} + \delta_2 \cdot \texttt{prisonerdeserves}_{i} + X^{\prime}_{i} \cdot \texttt{controlsvector} + \epsilon_{i}
\]

and on punitive policy

\[
\texttt{prisonpen}_{i} = \delta_0 + \delta_1 \cdot \texttt{impact}_{i} + \delta_2 \cdot \texttt{prisonerdeserves}_{i} + X^{\prime}_{i} \cdot \texttt{controlsvector} + \epsilon_{i}
\]

In each model, \(delta_1\) captures the association between carceral
contact and policy preferences, controlling for deservingness and
covariates. \(delta_2\) estimates the relationship between deservingness
and policy support, which directly tests H2. A positive and
statistically significant \(delta_2\) in the help model---and a negative
\(delta_2\) in the punitive model---would support the hypothesis that
warmer evaluations of incarcerated people correspond to greater support
for decarceral policies.

To formally test whether perceptions of deservingness mediate the effect
of carceral contact on policy preferences, I first estimate a set of
interaction models to examine whether the relationship between
deservingness and policy views differs depending on whether respondents
have been directly or proximally impacted by the criminal legal system.
This approach allows me to test whether contact moderates how
deservingness evaluations translate into policy preferences---i.e.,
whether deservingness matters more (or less) for those with lived
experience.

\subparagraph{Model 5: Interaction for Support of Carceral Policy
Help}\label{model-5-interaction-for-support-of-carceral-policy-help}

\[
\texttt{prisonhelp}_{i} = \theta_0 + \theta_1 \cdot \texttt{Deservingness}_{i} + \theta_2 \cdot \texttt{Contact}_{i} + \theta_3 (\texttt{Deservingness}_{i} \times \texttt{Contact}_{i}) + X^{\prime}_{i} \cdot \texttt{controlsvector} + \epsilon_{i}
\]

\subparagraph{Model 6: Interaction for Support of Carceral Policy
Punishment}\label{model-6-interaction-for-support-of-carceral-policy-punishment}

\[
\texttt{prisonpen}_{i} = \theta_0 + \theta_1 \cdot \texttt{Deservingness}_{i} + \theta_2 \cdot \texttt{Contact}_{i} + \theta_3 (\texttt{Deservingness}_{i} \times \texttt{Contact}_{i}) + X^{\prime}_{i} \cdot \texttt{controlsvector} + \epsilon_{i}
\]

A significant interaction term (\$\textbackslash theta\_3\$) would
indicate that the effect of deservingness on policy support varies
depending on whether the respondent has had contact with the criminal
legal system. These models help determine whether deservingness operates
similarly across groups---or if lived experience conditions how moral
evaluations shape policy preferences.

Building on the interaction models, I next, I assess whether
deservingness perceptions partially mediate the relationship between
contact and policy preferences. This directly follows from my
theoretical framework: if contact within the criminal legal system
influences how individuals perceive the deservingness of incarcerated
people, and if those perceptions in turn shape policy support,
deservingness should function as a mediating mechanism.

I estimate causal mediation using the `mediation' package in R,
estimating average indirect effects and confidence intervals via
bootstrapping. This involves first estimating a model in which
deservingness is regressed on contact, and then a second-stage model in
which policy preferences are regressed on both contact and
deservingness. These models include the same attitudinal and demographic
controls used throughout the analysis. The resulting estimates of the
average causal mediation effect (ACME) help evaluate whether contact
shapes policy views in part through its effect on perceptions of
deservingness.

For both the prison help and prison penalty indices, I estimate:

\subparagraph{Model 7: Mediation Outcome Model - Support for Assistance
Oriented Policy (Second Stage for Mediation
Analysis)}\label{model-7-mediation-outcome-model---support-for-assistance-oriented-policy-second-stage-for-mediation-analysis}

\[
y_{i}^{\texttt{prisonhelp}} = \beta_0 + \beta_1 \cdot \texttt{impact}_{i} + \beta_2 \cdot \texttt{des\_or}_{i} + X^{\prime}_{i} \cdot \texttt{controlsvector} + u_{i}
\]

\subparagraph{Model 8: Mediation Outcome Model - Support for Punishment
Oriented Policy (Second Stage for Mediation
Analysis)}\label{model-8-mediation-outcome-model---support-for-punishment-oriented-policy-second-stage-for-mediation-analysis}

\[
y_{i}^{\texttt{prisonpen}} = \beta_0 + \beta_1 \cdot \texttt{impact}_{i} + \beta_2 \cdot \texttt{des\_or}_{i} + X^{\prime}_{i} \cdot \texttt{controlsvector} + u_{i}
\]

where \(y_{i}^{\texttt{prisonhelp}}\) is support for assistance-oriented
or and \(y_{i}^{\texttt{prisonpen}}\) is support for punitive policies.
A statistically significant indirect effect would indicate that
deservingness mediates part of the relationship between carceral contact
and policy preferences.

I model criminal legal contact's effect on other forms of policy help by
regressing aggregated attitudes towards those policies on impact:

\subparagraph{Model 9: Robustness Check: General Policy Help Orientation
(OLS regression of composite non-carceral help policy support on
contact)}\label{model-9-robustness-check-general-policy-help-orientation-ols-regression-of-composite-non-carceral-help-policy-support-on-contact}

\[
y_{i}^{\texttt{polorientation}} = \beta_0 + \beta_1 \cdot \texttt{impact}_{i} + \beta_2 \cdot \texttt{des\_or}_{i} + X^{\prime}_{i} \cdot \texttt{controlsvector} + u_{i}
\]

To evaluate whether \textbf{(H2)} more generous perceptions of carceral
citizens' deservingness also predict more consistent \emph{consistent}
support for offering policy help to carceral citizens, I conduct an
additional analysis focusing on response consistency. First, I calculate
the variance of each respondent's responses across all six help-oriented
policy questions (with higher values indicating less consistent
responses). I then use this variance measure as the dependent variable
in a regression model with the same predictors as my main policy
analysis.

\subparagraph{Model 10: Explanatory Model: Variance in Support for
Assistance Oriented Policies (OLS regression of within-respondent
variance across decarceral or help policy
items)}\label{model-10-explanatory-model-variance-in-support-for-assistance-oriented-policies-ols-regression-of-within-respondent-variance-across-decarceral-or-help-policy-items}

\[
{\texttt{variancehelp}}_{i} = \beta_0 + \beta_1 \cdot {\texttt{impact}}_{i} + \beta_2 \cdot {\texttt{prisonerdeserves}}_{i} + \beta_3 \cdot {\texttt{polor}}_{i} + X_i^{\prime} \cdot {\texttt{controls}} + u_i
\]

Finally, to assess heterogeneity by contact type, I re-estimate the
above models using individual dummy variables for each form of impact
(e.g., been incarcerated, known someone, victim of crime). In
exploratory models, I restrict the sample to respondents who report only
one form of contact to reduce overlap and isolate effects.

\begin{center}\rule{0.5\linewidth}{0.5pt}\end{center}

\subsection{Conclusion}\label{conclusion}

This study investigates how carceral policy contact -- incarceration,
proximity to incarcerated individuals, or victimization -- shapes
perceptions of whether incarcerated people deserve assistance from the
state. To answer this question, this paper turns to the deservingness
framework. While only one paper has applied the deservingness framework
to the carceral domain,\footnote{One recent study applies the framework
  to criminal justice policy, though it takes a different empirical and
  theoretical approach than the one advanced here.} the deservingness
framework's broader use across welfare, health, and immigration
illustrates its adaptability to domains in which the moral logic of
state intervention---whether distributive, punitive, or
stigmatizing---differs in form and function. Across these contexts, the
framework has helped explain how individuals judge policy targets as
worthy or unworthy of state help, harm, or indifference.

To adapt deservingness to this setting and develop expectations, the
paper integrates insights from policy feedback scholarship. Here,
scholars document how lived experience with state policy shapes
political attitudes -- particular towards state legitimacy,
responsiveness, and one's civic standing. While policy feedback theory
has increasingly been applied to the carceral state, it has rarely
focused on how contact shapes interpretive judgements about policy
subjects -- particularly in regards to perceived worth or moral
standing. This paper contributes by asking: does experience with
criminal legal policy -- it's design, enforcement, and consequences --
change how people see those caught up in it, and the state intervention
they deserve?

Several possible findings, each theoretically consequential, follow from
the empirical design:

\begin{itemize}
\item
  If carceral contact increases perceived deservingness, this would
  build on recent contributions that social constructions of
  deservingness are not fixed, but contingent on policy context
  (Kreitzer and Smith 2018) and experience (SoRelle and Laws 2024b). It
  would suggest that, as Schneider and Ingram (2019) noted, feedback
  dynamics around ``deviant'' groups may be more complicated, that even
  in a ``sticky,'' stigmatized, punitive domain, interpretive shifts are
  possible, and that policy experience may soften exclusionary
  constructions. This would also extend the reach of policy feedback
  theory into moral-evaluative terrain. Future research, including
  experimental designs such as conjoint studies, could help identify
  what features of contact drive these changes---whether CARIN criteria,
  or others.
\item
  If deservingness mediates the relationship between contact and policy
  support, the findings would align with classic formulations of the
  deservingness framework: that people must first judge targets as
  worthy before they support redistributive or rehabilitative policy. It
  would underscore deservingness as a distinct mechanism---linking lived
  experience to moral evaluation, and moral evaluation to political
  preference---and suggest that deservingness continues to operate as a
  central schema even in domains marked by punishment rather than care.
  Further, it would raise important questions for what differences
  across impact groups mean against representational inequality and
  policy outcomes -- audit studies or legislative surveys or interviews
  would be useful for understanding how public perceptions of carceral
  deservingness drive policy outcomes and responsiveness.
\item
  Should contact lead to warmer policy views independent of perceived
  deservingness, it would raise further questions about the application
  of deservingness to the carceral domain. Such results might indicate
  that highly visible, invective, and emotive constructions of a target
  population still shape perceptions of that target's moral or deserving
  standing in society, but that individuals draw on other considerations
  -- such as procedural or institutional fairness, personalized or
  detailed understandings of carceral contexts and consequences, or
  empathy - when constructing criminal legal policy views.
\item
  Last, if contact yields no observable effects, this null result would
  still carry interpretive weight. It would suggest that carceral policy
  operates through durable, symbolic considerations that lived
  experience alone cannot unsettle.
\end{itemize}

This project also takes seriously the role of interpretation in shaping
the political consequences of policy contact. As scholars of policy
feedback have emphasized, the lessons people draw from state action
depend not only on institutional design, but on how individuals make
sense of their experiences. Deservingness, as used here and across
domains, is useful not only because it helps map policy views, but
because it helps illustrate that active process: how people categorize
others as blameworthy or redeemable, threatening or vulnerable, and how
those categorizations shape--and are shaped by--encounters with state
power. In asking whether experiences of the criminal legal system shift
how people view its targets, this paper responds to Wedeen's (2002, 720)
call for empirical accounts of ``how symbols operate in practice, why
meanings generate action, and why actions produce meanings, when they
do.'' Deservingness, in this process, is a politically meaningful
interpretation of who is owed what, by whom, and why.

. . .

\theendnotes

\subsection{References}\label{references}

\subsection{Appendix}\label{appendix}

\subsubsection{A: Variable Dictionary}\label{a-variable-dictionary}

\begin{longtable}[]{@{}
  >{\raggedright\arraybackslash}p{(\columnwidth - 4\tabcolsep) * \real{0.2778}}
  >{\raggedright\arraybackslash}p{(\columnwidth - 4\tabcolsep) * \real{0.2778}}
  >{\raggedright\arraybackslash}p{(\columnwidth - 4\tabcolsep) * \real{0.4444}}@{}}
\toprule\noalign{}
\begin{minipage}[b]{\linewidth}\raggedright
Variable Name
\end{minipage} & \begin{minipage}[b]{\linewidth}\raggedright
Type
\end{minipage} & \begin{minipage}[b]{\linewidth}\raggedright
Description
\end{minipage} \\
\midrule\noalign{}
\endhead
\bottomrule\noalign{}
\endlastfoot
\texttt{impact} & Binary & Any carceral or victimization contact \\
\texttt{incarc} & Binary & Has been incarcerated \\
\texttt{knowincarc} & Binary & Knows someone incarcerated \\
\texttt{vict} & Binary & Has been victim of violent crime \\
\texttt{prisonerdeserves} & Continuous & Centered deservingness of
incarcerated people \\
\texttt{prisonhelp} & Index & Support for prison-assistance policies \\
\texttt{prisonpen} & Index & Support for punitive carceral policies \\
\texttt{variance\_help} & Continuous & Variance across prison-help
policy support items \\
\texttt{sd\_help} & Continuous & Standard deviation of help-policy
support items \\
\texttt{polorientation} / \texttt{pol-or} & Index & General orientation
toward social policy help \\
\texttt{desor} & Continuous & Centered deservingness of help-receiving
groups \\
\texttt{Republican} & Dummy & Identifies as Republican \\
\texttt{Democrat} & Dummy & Identifies as Democrat \\
\texttt{ideology} & Index & Political ideology (0 = liberal, 1 =
conservative) \\
\texttt{Punitive} & Index & General punitive orientation \\
\texttt{PenalPunitiveness} & Ordinal / Rescaled & Beliefs about
punishment vs rehabilitation \\
\texttt{PolicyDisposition} & Index & Support for redistributive and
social help policies \\
\texttt{fire\_rare} & Likert (1--4) & Belief that racial problems are
rare and isolated \\
\texttt{fire\_privilege} & Likert (1--4) & Belief that white people
benefit from racial privilege \\
\texttt{fire\_angry} & Likert (1--4) & Anger about the existence of
racism \\
\texttt{fire\_fear} & Likert (1--4) & Fear of people of other races \\
\texttt{gender} & Dummy & 0 = woman/other, 1 = man \\
\texttt{education} & Ordinal & Educational attainment level \\
\texttt{Black}, \texttt{Hispanic}, \texttt{OtherRace} & Dummy & Racial
group identifiers \\
\texttt{impact\ ×\ ideology} & Interaction & Tests moderation by
ideology \\
\texttt{impact\ ×\ race} & Interaction & Tests moderation by race \\
\texttt{Deservingness\ ×\ Contact} & Interaction & Moderation of
deservingness effect \\
\texttt{X\_\{i\}\textquotesingle{}\ x\ controls} & Vector & Full vector
of demographic and attitudinal controls \\
\end{longtable}

\subsubsection{Appendix B: Robustness Checks and Conditional
Effects}\label{appendix-b-robustness-checks-and-conditional-effects}

This section reports additional models assessing whether the
relationship between contact, deservingness, and policy views varies
across key individual-level characteristics. Although not central to my
theoretical framework, these checks address common concerns that
ideological predispositions or social group membership may condition the
observed effects.

Although not a core hypothesis, I explore whether the effect of carceral
contact on deservingness varies across ideological groups. This analysis
addresses concerns that the relationship between contact and
deservingness may simply reflect underlying political ideology. More
broadly, prior work suggests that while ideology is a powerful predictor
of attitudes toward marginalized groups, it may also obscure important
conditional variation in how individuals interpret personal or vicarious
experiences with the criminal legal system.\footnote{While general
  ideology can be blunt instruments for understanding attitudes in
  complex or racialized policy domains (Kreitzer, Maltby, and Smith
  2022), and while partisans may agree on many group evaluations,
  intra-partisan disagreements reflect deep symbolic divides with
  important implications for policy design (Kreitzer and Smith 2024).}
Thus, I test the possibility that respondents with different ideological
orientations may interpret contact differently. Specifically, I
estimate:

\[
\texttt{prisonerdeserves}_{i} = \beta_0 + \beta_1 \cdot \texttt{impact}_{i} + \beta_2 \cdot \texttt{gender}_{i} + \beta_3 \cdot \texttt{race}_{i} + \beta_4 \cdot \texttt{education}_{i} + \beta_5 \cdot \texttt{urban}_{i} + \beta_6 \cdot \texttt{party}_{i} + \beta_7 \cdot \texttt{ideology}_{i} + \beta_8 \cdot \texttt{racialresentment}_{i} + \beta_9 \cdot \texttt{desor}_{i} + \beta_{10} \cdot \texttt{policydisposition}_{i} + \beta_{11} \cdot (\texttt{impact}_{i} \times \texttt{ideology}_{i}) + \beta_{12} \cdot (\texttt{impact}_{i} \times \texttt{race}_{i}) + u_{i}
\]

A positive and statistically significant \(\beta_3\) would indicate that
contact increases deservingness more among liberal respondents than
conservative ones.

In an exploratory analysis, I examine whether the relationship between
criminal legal contact and deservingness differs by respondent race and
racial attitudes. Although not predicted by my core theoretical
framework, this test allows for the possibility that lived experience
and racialized constructions of criminality may moderate how individuals
interpret contact. Race is central to matters of carceral policy
(Michener 2019) as are racial ``attitudes'' to punitiveness in the
American context (Fresh, n.d.). I estimate the following model:

\[
\text{prisonerdeserves}_i = \beta_0 + \beta_1 \cdot \text{impact}_i + \beta_2 \cdot \text{race}_i + \beta_3 \cdot (\text{impact}_i \times \text{race}_i) + \beta_4 \cdot \text{FIRE}_i + \mathbf{X}_i' \boldsymbol{\beta} + u_i
\]

Although presented in simplified form above, each component of the model
is entered separately in estimation. Specifically, \(\text{race}_i\)
represents a set of three mutually exclusive indicator variables for
racial self-identification (non-Hispanic Black, Hispanic/Latino/a/e, and
``Other,'' with non-Hispanic white as the reference category). The
interaction term \(\text{impact}_i \times \text{race}_i\) refers to
separate interaction terms between carceral contact and each racial
group. \(\text{FIRE}_i\) refers to the four-item battery capturing
racial fear, institutional racism acknowledgment, and racial empathy,
with each item entered individually to preserve its theoretical
distinctiveness. \(\mathbf{X}_i'\) includes standard demographic and
attitudinal controls.

\subsubsection{Appendix B: Exploratory Models of Policy Response
Consistency}\label{appendix-b-exploratory-models-of-policy-response-consistency}

While my primary tests of H2 focus on average support for assistance-
and penalty-oriented policies, this section explores whether criminal
legal contact is also associated with greater consistency in support
across multiple decarceral policy items. Social construction theory
suggests that personal connection to stigmatized policy targets may
reduce reliance on stereotypes, potentially leading to more coherent or
unconditional policy preferences. The models below assess this
possibility by estimating the variance and standard deviation of
respondents' support for help-oriented policies, as well as fitting a
mixed effects model to evaluate within-respondent consistency.

To confirm my findings with an alternative measure of consistency, I
also calculate the standard deviation of each person's responses across
the already standardized help policy items. This provides a normalized
measure of response dispersion:

\[
\texttt{sdhelp}_{i} = \beta_0 + \beta_1 \cdot \texttt{impact}_{i} + \beta_2 \cdot \texttt{prisonerdeserves}_{i} + \beta_3 \cdot \texttt{polor}_{i} + X^{\prime}_{i} \cdot \texttt{controls} + u_{i}
\]

I expect that \(\beta_1\) will be negative in both models, indicating
that individuals with direct or proximate experiences with incarceration
(impact =1) show more consistent support across different decarceral
policies, even after controlling for deservingness attitudes and other
demographic factors. This would support my hypothesis that personal
connection leads to more coherent or unconditional policy preferences,
rather than simply higher average support.

Because aggregate measures may obscure underlying variation, I fit an
ordinal logistic mixed effects model predicting policy support, with
random intercepts for respondents. The model includes fixed effects for
policy experience (impact), deservingness perceptions, attitudinal and
remaining controls:

{[} \begin{align}
\text{logit}\left( \Pr(Y_{ij} \leq k) \right)
&= \alpha_k - \Big(
\beta_1 \cdot \text{impact}_i +
\beta_2 \cdot \text{prisonerdeserves}_i +
\beta_3 \cdot \text{pol\_or}_i +
\mathbf{X}_i^{\prime} \boldsymbol{\beta} +
u_i + \varepsilon_{ij}
\Big)
\end{align} {]}

Here, \(\beta_1\) estimates the average effect of contact on policy
preferences; \(\beta_2\) captures how much perceived deservingness of
incarcerated people influences policy support; and \(\beta_3\) reflects
policy help attitudes. The term
\(\mathbf{X}_{i}^{\prime}\boldsymbol{\beta}\) represents control
covariates, while \(u_{i}\) allows for respondent-specific baseline
support levels. This enables me to model individual-level responses
across all policy items simultaneously, and assess both the average
effect of experience on policy support (through the fixed effect
coefficient) and the consistency of support across policies (through the
random effects variance).

\phantomsection\label{refs}
\begin{CSLReferences}{1}{0}
\bibitem[\citeproctext]{ref-anoll2022}
Anoll, Allison P., Derek A. Epp, and Mackenzie Israel-Trummel. 2022a.
{``Contact and Context: How Municipal Traffic Stops Shape Citizen
Character.''} \emph{The Journal of Politics} 84 (4): 2272--77.
\url{https://doi.org/10.1086/719274}.

\bibitem[\citeproctext]{ref-anoll2022a}
---------. 2022b. {``Contact and Context: How Municipal Traffic Stops
Shape Citizen Character.''} \emph{The Journal of Politics} 84 (4):
2272--77. \url{https://doi.org/10.1086/719274}.

\bibitem[\citeproctext]{ref-anoll2019}
Anoll, Allison, and Mackenzie Israel-Trummel. 2019a. {``Do Felony
Disenfranchisement Laws (De)Mobilize? A Case of Surrogate
Participation.''} \emph{The Journal of Politics} 81 (4): 1523--27.
\url{https://doi.org/10.1086/704783}.

\bibitem[\citeproctext]{ref-anoll2019a}
---------. 2019b. {``Do Felony Disenfranchisement Laws (De)Mobilize? A
Case of Surrogate Participation.''} \emph{The Journal of Politics} 81
(4): 1523--27. \url{https://doi.org/10.1086/704783}.

\bibitem[\citeproctext]{ref-burch2014}
Burch, Traci R. 2014. {``Effects of Imprisonment and Community
Supervision on Neighborhood Political Participation in North
Carolina.''} \emph{The ANNALS of the American Academy of Political and
Social Science} 651 (1): 184--201.
\url{https://doi.org/10.1177/0002716213503093}.

\bibitem[\citeproctext]{ref-campbell2011}
Campbell, Andrea Louise. 2011. \emph{How Policies Make Citizens: Senior
Political Activism and the American Welfare State}. Princeton Studies in
American Politics: Historical, International, and Comparative
Perspectives. Princeton: Princeton University Press.

\bibitem[\citeproctext]{ref-campbell2012}
---------. 2012. {``Policy Makes Mass Politics.''} \emph{Annual Review
of Political Science} 15 (1): 333--51.
\url{https://doi.org/10.1146/annurev-polisci-012610-135202}.

\bibitem[\citeproctext]{ref-culhane2004}
Culhane, Scott E., Harmon M. Hosch, and William G. Weaver. 2004.
{``Crime Victims Serving as Jurors: Is There Bias Present?''} \emph{Law
and Human Behavior} 28 (6): 649--59.
\url{https://doi.org/10.1007/s10979-004-0792-1}.

\bibitem[\citeproctext]{ref-desante2013}
DeSante, Christopher D. 2013. {``Working Twice as Hard to Get Half as
Far: Race, Work Ethic, and America{'}s Deserving Poor:
{\emph{DESERVINGNESS IN BLACK AND WHITE}}.''} \emph{American Journal of
Political Science} 57 (2): 342--56.
\url{https://doi.org/10.1111/ajps.12006}.

\bibitem[\citeproctext]{ref-desante2020}
DeSante, Christopher D., and Candis Watts Smith. 2020. {``Fear,
Institutionalized Racism, and Empathy: The Underlying Dimensions of
Whites{'} Racial Attitudes.''} \emph{PS: Political Science \& Politics}
53 (4): 639--45. \url{https://doi.org/10.1017/S1049096520000414}.

\bibitem[\citeproctext]{ref-ditton1999}
Ditton, Jason, Jon Bannister, Elizabeth Gilchrist, and Stephen Farrall.
1999. {``Afraid or Angry? Recalibrating the {`}Fear{'} of Crime.''}
\emph{International Review of Victimology} 6 (2): 83--99.
\url{https://doi.org/10.1177/026975809900600201}.

\bibitem[\citeproctext]{ref-donovan2007}
Donovan, Roxanne A. 2007. {``To Blame or Not To Blame: Influences of
Target Race and Observer Sex on Rape Blame Attribution.''} \emph{Journal
of Interpersonal Violence} 22 (6): 722--36.
\url{https://doi.org/10.1177/0886260507300754}.

\bibitem[\citeproctext]{ref-ellis2020a}
Ellis, Christopher, and Christopher Faricy. 2020. {``Race,
{``}Deservingness,{''} and Social Spending Attitudes: The Role of Policy
Delivery Mechanism.''} \emph{Political Behavior} 42 (3): 819--43.
\url{https://doi.org/10.1007/s11109-018-09521-w}.

\bibitem[\citeproctext]{ref-ewald2024}
Ewald, Alec C. 2024. {``{``}Because I Feel Like I Want to Be Heard, You
Know?:{''} Carceral Citizenship and Collateral Consequences.''}
\emph{Law \& Policy} 46 (1): 4--26.
\url{https://doi.org/10.1111/lapo.12225}.

\bibitem[\citeproctext]{ref-feather2009a}
Feather, N. T., and Ian R. and McKee. 2009. {``Differentiating Emotions
in Relation to Deserved or Undeserved Outcomes: A Retrospective Study of
Real-Life Events.''} \emph{Cognition and Emotion} 23 (5): 955--77.
\url{https://doi.org/10.1080/02699930802243378}.

\bibitem[\citeproctext]{ref-fresh}
Fresh, Adriane. n.d. {``The Racial Geography of U.S. Public Opinion at
the Punitive Turn.''}

\bibitem[\citeproctext]{ref-ghandnoosh2014}
Ghandnoosh, Nazgol. 2014. {``Race and Punishment: Racial Perceptions of
Crime and Support for Punitive Policies.''}
\url{https://www.sentencingproject.org/reports/race-and-punishment-racial-perceptions-of-crime-and-support-for-punitive-policies/}.

\bibitem[\citeproctext]{ref-gollust2011}
Gollust, Sarah E., and Julia Lynch. 2011. {``Who Deserves Health Care?
The Effects of Causal Attributions and Group Cues on Public Attitudes
about Responsibility for Health Care Costs.''} \emph{Journal of Health
Politics, Policy and Law} 36 (6): 1061--95.
\url{https://doi.org/10.1215/03616878-1460578}.

\bibitem[\citeproctext]{ref-goss2012}
Goss, Kristin. 2012. \emph{The Paradox of Gender Equality: How American
Women's Groups Gained and Lost Their Public Voice}. Ann Arbor, MI:
University of Michigan Press.
\url{https://doi.org/10.3998/mpub.4844961}.

\bibitem[\citeproctext]{ref-hale1996}
Hale, C. 1996. {``Fear of Crime: A Review of the Literature.''}
\emph{International Review of Victimology} 4 (2): 79150.
\url{https://doi.org/10.1177/026975809600400201}.

\bibitem[\citeproctext]{ref-israel-trummel2022}
Israel-Trummel, Mackenzie, and Shea Streeter. 2022. {``Police Abuse or
Just Deserts?''} \emph{Public Opinion Quarterly} 86 (S1): 499--522.
\url{https://doi.org/10.1093/poq/nfac017}.

\bibitem[\citeproctext]{ref-jennings2012}
Jennings, Wesley G., Alex R. Piquero, and Jennifer M. Reingle. 2012.
{``On the Overlap Between Victimization and Offending: A Review of the
Literature.''} \emph{Aggression and Violent Behavior} 17 (1): 16--26.
\url{https://doi.org/10.1016/j.avb.2011.09.003}.

\bibitem[\citeproctext]{ref-kleck2017}
Kleck, Gary, and Dylan Baker Jackson. 2017b. {``Does Crime Cause
Punitiveness?''} \emph{Crime \& Delinquency} 63 (12): 1572--99.
\url{https://doi.org/10.1177/0011128716638503}.

\bibitem[\citeproctext]{ref-kleck2017a}
---------. 2017a. {``Does Crime Cause Punitiveness?''} \emph{Crime \&
Delinquency} 63 (12): 1572--99.
\url{https://doi.org/10.1177/0011128716638503}.

\bibitem[\citeproctext]{ref-kreitzer2022}
Kreitzer, Rebecca J., Elizabeth A. Maltby, and Candis Watts Smith.
2022a. {``Fifty Shades of Deservingness: An Analysis of State-Level
Variation and Effect of Social Constructions on Policy Outcomes.''}
\emph{Journal of Public Policy} 42 (3): 436--64.
\url{https://doi.org/10.1017/S0143814X21000222}.

\bibitem[\citeproctext]{ref-kreitzer2022a}
---------. 2022b. {``Fifty Shades of Deservingness: An Analysis of
State-Level Variation and Effect of Social Constructions on Policy
Outcomes.''} \emph{Journal of Public Policy} 42 (3): 436--64.
\url{https://doi.org/10.1017/S0143814X21000222}.

\bibitem[\citeproctext]{ref-kreitzer2018}
Kreitzer, Rebecca J., and Candis Watts Smith. 2018. {``Reproducible and
Replicable: An Empirical Assessment of the Social Construction of
Politically Relevant Target Groups.''} \emph{PS: Political Science \&
Politics} 51 (4): 768--74.
\url{https://doi.org/10.1017/S1049096518000987}.

\bibitem[\citeproctext]{ref-lacombe2022}
Lacombe, Matthew J. 2022. {``Post-Loss Power Building: The Feedback
Effects of Policy Loss on Group Identity and Collective Action.''}
\emph{Policy Studies Journal} 50 (3): 507--26.
\url{https://doi.org/10.1111/psj.12446}.

\bibitem[\citeproctext]{ref-lee2014}
Lee, Hedwig, Lauren C. Porter, and Megan Comfort. 2014. {``Consequences
of Family Member Incarceration: Impacts on Civic Participation and
Perceptions of the Legitimacy and Fairness of Government.''} \emph{The
ANNALS of the American Academy of Political and Social Science} 651 (1):
44--73. \url{https://doi.org/10.1177/0002716213502920}.

\bibitem[\citeproctext]{ref-lensa}
Lensa, Kim M E, Janne van Doornb, Antony Pembertona, and Stefan
Bogaertsa. n.d. {``You Shouldn't Feel That Way! Extending the Emotional
Victim Effect Through the Mediating Role of Expectancy Violation.''}

\bibitem[\citeproctext]{ref-lerman2014}
Lerman, Amy E., and Vesla Weaver. 2014a. \emph{Arresting Citizenship:
The Democratic Consequences of American Crime Control}. Univesity of
Chicago Press.

\bibitem[\citeproctext]{ref-lerman2014a}
---------. 2014c. \emph{Arresting Citizenship: The Democratic
Consequences of American Crime Control}. Univesity of Chicago Press.

\bibitem[\citeproctext]{ref-lerman2014b}
---------. 2014b. \emph{Arresting Citizenship: The Democratic
Consequences of American Crime Control}. Univesity of Chicago Press.

\bibitem[\citeproctext]{ref-levine}
Levine, Jeremy R, and Kelly L Russell. n.d. {``Crime Pays the Victim:
Criminal Fines, the State, and Victim Compensation Law
1964{\textendash}1984.''} \emph{American Journal of Sociology}.

\bibitem[\citeproctext]{ref-levinson2007}
Levinson, Justin D. 2007. {``Forgotten Racial Equality: Implicit Bias,
Decisionmaking, and Misremembering.''} \emph{Duke Law Journal} 57 (2):
345--424.
\url{https://heinonline.org/HOL/P?h=hein.journals/duklr57/&i=349}.

\bibitem[\citeproctext]{ref-levinson2010}
Levinson, Justin D., Huajian Cai, and Danielle Young. 2010. {``Guilty by
Implicit Racial Bias: The Guilty/Not Guilty Implicit Association Test
Commentaries.''} \emph{Ohio State Journal of Criminal Law} 8 (1):
187--208.
\url{https://heinonline.org/HOL/P?h=hein.journals/osjcl8/&i=189}.

\bibitem[\citeproctext]{ref-mallorysorelle2022}
Mallory SoRelle, and Serena Laws. 2022. {``Deservingness and the
Politics of Student Debt Relief.''} \emph{Working Paper}.

\bibitem[\citeproctext]{ref-maltby2023}
Maltby, Elizabeth, and Rebecca J. Kreitzer. 2023. {``How Racialized
Policy Contact Shapes the Social Constructions of Policy Targets.''}
\emph{Policy Studies Journal} 51 (1): 145--62.
\url{https://doi.org/10.1111/psj.12481}.

\bibitem[\citeproctext]{ref-manza2006}
Manza, Jeff, Christopher Uggen, and Clem Brooks. 2006. {``Public Opinion
and Felon Disenfranchisement.''} In, edited by Jeff Manza and
Christopher Uggen, 0. Oxford University Press.
\url{https://doi.org/10.1093/acprof:oso/9780195149326.003.0050}.

\bibitem[\citeproctext]{ref-mettler2004}
Mettler, Suzanne, and Joe Soss. 2004. {``The Consequences of Public
Policy for Democratic Citizenship: Bridging Policy Studies and Mass
Politics.''} \emph{Perspectives on Politics} 2 (1): 55--73.
\url{https://doi.org/10.1017/S1537592704000623}.

\bibitem[\citeproctext]{ref-michener2018}
Michener, Jamila. 2018. \emph{Fragmented Democracy: Medicaid,
Federalism, and Unequal Politics}. Cambridge: Cambridge University
Press.

\bibitem[\citeproctext]{ref-michener2019}
---------. 2019. {``Policy Feedback in a Racialized Polity.''}
\emph{Policy Studies Journal} 47 (2): 423--50.
\url{https://doi.org/10.1111/psj.12328}.

\bibitem[\citeproctext]{ref-oorschot2000}
Oorschot, Wim van. 2000a. {``Who Should Get What, and Why? On
Deservingness Criteria and the Conditionality of Solidarity Among the
Public.''} \emph{Policy \& Politics} 28 (1): 33--48.
\url{https://doi.org/10.1332/0305573002500811}.

\bibitem[\citeproctext]{ref-oorschot2000b}
---------. 2000b. {``Who Should Get What, and Why? On Deservingness
Criteria and the Conditionality of Solidarity Among the Public.''}
\emph{Policy \& Politics} 28 (1): 33--48.
\url{https://doi.org/10.1332/0305573002500811}.

\bibitem[\citeproctext]{ref-vanoorschot2006}
Oorschot, Wim van. 2006a. {``Making the Difference in Social Europe:
Deservingness Perceptions Among Citizens of European Welfare States.''}
\emph{Journal of European Social Policy} 16 (1): 23--42.
\url{https://doi.org/10.1177/0958928706059829}.

\bibitem[\citeproctext]{ref-vanoorschot2006b}
---------. 2006b. {``Making the Difference in Social Europe:
Deservingness Perceptions Among Citizens of European Welfare States.''}
\emph{Journal of European Social Policy} 16 (1): 23--42.
\url{https://doi.org/10.1177/0958928706059829}.

\bibitem[\citeproctext]{ref-owens2014}
Owens, Michael Leo. 2014. {``Ex-Felons{'} Organization-Based Political
Work for Carceral Reforms.''} \emph{The ANNALS of the American Academy
of Political and Social Science} 651 (1): 256--65.
\url{https://doi.org/10.1177/0002716213502933}.

\bibitem[\citeproctext]{ref-petersen2012}
Petersen, Michael Bang, Daniel Sznycer, Leda Cosmides, and John Tooby.
2012. {``Who Deserves Help? Evolutionary Psychology, Social Emotions,
and Public Opinion about Welfare.''} \emph{Political Psychology} 33 (3):
395--418. \url{https://doi.org/10.1111/j.1467-9221.2012.00883.x}.

\bibitem[\citeproctext]{ref-pierson1993a}
Pierson, Paul. 1993. {``When Effect Becomes Cause: Policy Feedback and
Political Change.''} \emph{World Politics} 45 (4): 595628.

\bibitem[\citeproctext]{ref-schneider2019}
Schneider, Anne L., and Helen M. Ingram. 2019. {``Social Constructions,
Anticipatory Feedback Strategies, and Deceptive Public Policy.''}
\emph{Policy Studies Journal} 47 (2): 206--36.
\url{https://doi.org/10.1111/psj.12281}.

\bibitem[\citeproctext]{ref-schneider1993}
Schneider, Anne, and Helen Ingram. 1993a. {``Social Construction of
Target Populations: Implications for Politics and Policy.''}
\emph{American Political Science Review} 87 (2): 334--47.
\url{https://doi.org/10.2307/2939044}.

\bibitem[\citeproctext]{ref-schneider1993a}
---------. 1993b. {``Social Construction of Target Populations:
Implications for Politics and Policy.''} \emph{American Political
Science Review} 87 (2): 334--47. \url{https://doi.org/10.2307/2939044}.

\bibitem[\citeproctext]{ref-skocpol1992}
Skocpol, Theda. 1992. \emph{Protecting Soldiers and Mothers : The
Political Origins of Social Policy in the United States}. Harvard
University Press.
\url{https://web-p-ebscohost-com.libezproxy2.syr.edu/ehost/ebookviewer/ebook/ZTAwMHhuYV9fMjgyNzIyX19BTg2?sid=f03c7422-117d-4b27-8acb-14159a4d79d4@redis&vid=0&format=EB&rid=1}.

\bibitem[\citeproctext]{ref-smith2024}
Smith, Candis Watts, and Rebecca J. Kreitzer. 2024. {``Where Is the
Party in Social Construction Theory? Partisan Mappings of Politically
Relevant Target Groups.''} \emph{The Journal of Politics} 86 (2):
624--41. \url{https://doi.org/10.1086/726956}.

\bibitem[\citeproctext]{ref-sorelle2020}
SoRelle, Mallory E. 2020. \emph{Democracy Declined: The Failed Politics
of Consumer Financial Protection}. Chicago Studies in American Politics.
Chicago ; London: The University of Chicago Press.

\bibitem[\citeproctext]{ref-sorelle2024}
SoRelle, Mallory E., and Serena Laws. 2024a. {``Deservingness and the
Politics of Student Debt Relief.''} \emph{Perspectives on Politics} 22
(2): 372--90. \url{https://doi.org/10.1017/S1537592723001457}.

\bibitem[\citeproctext]{ref-sorelle2024a}
---------. 2024b. {``Deservingness and the Politics of Student Debt
Relief.''} \emph{Perspectives on Politics} 22 (2): 372--90.
\url{https://doi.org/10.1017/S1537592723001457}.

\bibitem[\citeproctext]{ref-soss1999}
Soss, Joe. 1999a. {``Lessons of Welfare: Policy Design, Political
Learning, and Political Action.''} \emph{American Political Science
Review} 93 (2): 363--80. \url{https://doi.org/10.2307/2585401}.

\bibitem[\citeproctext]{ref-soss1999a}
---------. 1999b. {``Lessons of Welfare: Policy Design, Political
Learning, and Political Action.''} \emph{American Political Science
Review} 93 (2): 363--80. \url{https://doi.org/10.2307/2585401}.

\bibitem[\citeproctext]{ref-soss1999b}
---------. 1999c. {``Lessons of Welfare: Policy Design, Political
Learning, and Political Action.''} \emph{American Political Science
Review} 93 (2): 363--80. \url{https://doi.org/10.2307/2585401}.

\bibitem[\citeproctext]{ref-thurston2018}
Thurston, Chloe N. 2018. \emph{At the Boundaries of Homeownership:
Credit, Discrimination, and the American State}. 1st ed. New York, NY:
Cambridge University Press.

\bibitem[\citeproctext]{ref-walker2014}
Walker, Hannah L. 2014. {``Extending the Effects of the Carceral
State.''} \emph{Political Research Quarterly} 67 (4): 809--22.
\url{https://doi.org/10.1177/1065912914542522}.

\bibitem[\citeproctext]{ref-walker2020a}
---------. 2020a. \emph{Mobilized by Injustice: Criminal Justice
Contact, Political Participation, and Race}. Oxford University Press.
\url{https://academic.oup.com/book/32150}.

\bibitem[\citeproctext]{ref-walker2020}
---------. 2020b. {``Targeted: The Mobilizing Effect of Perceptions of
Unfair Policing Practices.''} \emph{The Journal of Politics} 82 (1):
119--34. \url{https://doi.org/10.1086/705684}.

\bibitem[\citeproctext]{ref-walker2017}
Walker, Hannah L., and Marcela García-Castañon. 2017. {``For Love and
Justice: The Mobilizing of Race, Gender, and Criminal Justice
Contact.''} \emph{Politics \& Gender} 13 (4): 541--68.
\url{https://doi.org/10.1017/S1743923X17000198}.

\bibitem[\citeproctext]{ref-walker2016}
Walker, Michael L. 2016. {``Race Making in a Penal Institution.''}
\emph{American Journal of Sociology} 121 (4): 1051--78.
\url{https://www.jstor.org/stable/26545705}.

\bibitem[\citeproctext]{ref-weaver2010}
Weaver, Vesla M., and Amy E. Lerman. 2010a. {``Political Consequences of
the Carceral State.''} \emph{American Political Science Review} 104 (4):
817--33. \url{https://doi.org/10.1017/S0003055410000456}.

\bibitem[\citeproctext]{ref-weaver2010a}
---------. 2010b. {``Political Consequences of the Carceral State.''}
\emph{American Political Science Review} 104 (4): 817--33.
\url{https://doi.org/10.1017/S0003055410000456}.

\bibitem[\citeproctext]{ref-weaver2010b}
---------. 2010c. {``Political Consequences of the Carceral State.''}
\emph{American Political Science Review} 104 (4): 817--33.
\url{https://doi.org/10.1017/S0003055410000456}.

\bibitem[\citeproctext]{ref-mettler2023a}
Weible, Christopher M. 2023. {``Policy Feedback Theory.''} In, 5th ed.,
100--129. New York: Routledge.
\url{https://doi.org/10.4324/9781003308201-5}.

\bibitem[\citeproctext]{ref-white2019}
White, Ariel. 2019. {``Misdemeanor Disenfranchisement? The Demobilizing
Effects of Brief Jail Spells on Potential Voters.''} \emph{American
Political Science Review} 113 (2): 311--24.
\url{https://doi.org/10.1017/S000305541800093X}.

\end{CSLReferences}



\end{document}

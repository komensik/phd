% Options for packages loaded elsewhere
% Options for packages loaded elsewhere
\PassOptionsToPackage{unicode}{hyperref}
\PassOptionsToPackage{hyphens}{url}
\PassOptionsToPackage{dvipsnames,svgnames,x11names}{xcolor}
%
\documentclass[
  letterpaper,
  DIV=11,
  numbers=noendperiod]{scrartcl}
\usepackage{xcolor}
\usepackage{amsmath,amssymb}
\setcounter{secnumdepth}{5}
\usepackage{iftex}
\ifPDFTeX
  \usepackage[T1]{fontenc}
  \usepackage[utf8]{inputenc}
  \usepackage{textcomp} % provide euro and other symbols
\else % if luatex or xetex
  \usepackage{unicode-math} % this also loads fontspec
  \defaultfontfeatures{Scale=MatchLowercase}
  \defaultfontfeatures[\rmfamily]{Ligatures=TeX,Scale=1}
\fi
\usepackage{lmodern}
\ifPDFTeX\else
  % xetex/luatex font selection
\fi
% Use upquote if available, for straight quotes in verbatim environments
\IfFileExists{upquote.sty}{\usepackage{upquote}}{}
\IfFileExists{microtype.sty}{% use microtype if available
  \usepackage[]{microtype}
  \UseMicrotypeSet[protrusion]{basicmath} % disable protrusion for tt fonts
}{}
\makeatletter
\@ifundefined{KOMAClassName}{% if non-KOMA class
  \IfFileExists{parskip.sty}{%
    \usepackage{parskip}
  }{% else
    \setlength{\parindent}{0pt}
    \setlength{\parskip}{6pt plus 2pt minus 1pt}}
}{% if KOMA class
  \KOMAoptions{parskip=half}}
\makeatother
% Make \paragraph and \subparagraph free-standing
\makeatletter
\ifx\paragraph\undefined\else
  \let\oldparagraph\paragraph
  \renewcommand{\paragraph}{
    \@ifstar
      \xxxParagraphStar
      \xxxParagraphNoStar
  }
  \newcommand{\xxxParagraphStar}[1]{\oldparagraph*{#1}\mbox{}}
  \newcommand{\xxxParagraphNoStar}[1]{\oldparagraph{#1}\mbox{}}
\fi
\ifx\subparagraph\undefined\else
  \let\oldsubparagraph\subparagraph
  \renewcommand{\subparagraph}{
    \@ifstar
      \xxxSubParagraphStar
      \xxxSubParagraphNoStar
  }
  \newcommand{\xxxSubParagraphStar}[1]{\oldsubparagraph*{#1}\mbox{}}
  \newcommand{\xxxSubParagraphNoStar}[1]{\oldsubparagraph{#1}\mbox{}}
\fi
\makeatother


\usepackage{longtable,booktabs,array}
\usepackage{calc} % for calculating minipage widths
% Correct order of tables after \paragraph or \subparagraph
\usepackage{etoolbox}
\makeatletter
\patchcmd\longtable{\par}{\if@noskipsec\mbox{}\fi\par}{}{}
\makeatother
% Allow footnotes in longtable head/foot
\IfFileExists{footnotehyper.sty}{\usepackage{footnotehyper}}{\usepackage{footnote}}
\makesavenoteenv{longtable}
\usepackage{graphicx}
\makeatletter
\newsavebox\pandoc@box
\newcommand*\pandocbounded[1]{% scales image to fit in text height/width
  \sbox\pandoc@box{#1}%
  \Gscale@div\@tempa{\textheight}{\dimexpr\ht\pandoc@box+\dp\pandoc@box\relax}%
  \Gscale@div\@tempb{\linewidth}{\wd\pandoc@box}%
  \ifdim\@tempb\p@<\@tempa\p@\let\@tempa\@tempb\fi% select the smaller of both
  \ifdim\@tempa\p@<\p@\scalebox{\@tempa}{\usebox\pandoc@box}%
  \else\usebox{\pandoc@box}%
  \fi%
}
% Set default figure placement to htbp
\def\fps@figure{htbp}
\makeatother





\setlength{\emergencystretch}{3em} % prevent overfull lines

\providecommand{\tightlist}{%
  \setlength{\itemsep}{0pt}\setlength{\parskip}{0pt}}



 


\usepackage{booktabs}
\usepackage{longtable}
\usepackage{array}
\usepackage{multirow}
\usepackage{wrapfig}
\usepackage{float}
\usepackage{colortbl}
\usepackage{pdflscape}
\usepackage{tabu}
\usepackage{threeparttable}
\usepackage{threeparttablex}
\usepackage[normalem]{ulem}
\usepackage{makecell}
\usepackage{xcolor}
\KOMAoption{captions}{tableheading}
\makeatletter
\@ifpackageloaded{caption}{}{\usepackage{caption}}
\AtBeginDocument{%
\ifdefined\contentsname
  \renewcommand*\contentsname{Table of contents}
\else
  \newcommand\contentsname{Table of contents}
\fi
\ifdefined\listfigurename
  \renewcommand*\listfigurename{List of Figures}
\else
  \newcommand\listfigurename{List of Figures}
\fi
\ifdefined\listtablename
  \renewcommand*\listtablename{List of Tables}
\else
  \newcommand\listtablename{List of Tables}
\fi
\ifdefined\figurename
  \renewcommand*\figurename{Figure}
\else
  \newcommand\figurename{Figure}
\fi
\ifdefined\tablename
  \renewcommand*\tablename{Table}
\else
  \newcommand\tablename{Table}
\fi
}
\@ifpackageloaded{float}{}{\usepackage{float}}
\floatstyle{ruled}
\@ifundefined{c@chapter}{\newfloat{codelisting}{h}{lop}}{\newfloat{codelisting}{h}{lop}[chapter]}
\floatname{codelisting}{Listing}
\newcommand*\listoflistings{\listof{codelisting}{List of Listings}}
\makeatother
\makeatletter
\makeatother
\makeatletter
\@ifpackageloaded{caption}{}{\usepackage{caption}}
\@ifpackageloaded{subcaption}{}{\usepackage{subcaption}}
\makeatother
\usepackage{bookmark}
\IfFileExists{xurl.sty}{\usepackage{xurl}}{} % add URL line breaks if available
\urlstyle{same}
\hypersetup{
  pdftitle={p1\_results},
  colorlinks=true,
  linkcolor={blue},
  filecolor={Maroon},
  citecolor={Blue},
  urlcolor={Blue},
  pdfcreator={LaTeX via pandoc}}


\title{p1\_results}
\author{}
\date{}
\begin{document}
\maketitle

\renewcommand*\contentsname{Table of contents}
{
\hypersetup{linkcolor=}
\setcounter{tocdepth}{3}
\tableofcontents
}

\section{Results: Diagnostic Analyses of Contact and
Deservingness}\label{results-diagnostic-analyses-of-contact-and-deservingness}

\subsection{Overview}\label{overview}

These results document the exploratory and diagnostic tests used to
evaluate how distinct forms of contact with the criminal legal
system---\textbf{incarceration experience (\texttt{incarc})},
\textbf{knowing someone incarcerated (\texttt{knowincarc})}, and
\textbf{victimization (\texttt{vict})}---relate to perceived
deservingness of incarcerated people (\texttt{d\_incppl\_pris\_abc}).\\
Analyses are \textbf{unweighted} and focus on data integrity, model
performance, and variable behavior rather than publication-ready
estimates.

\subsection{I. Descriptive
Diagnostics}\label{i.-descriptive-diagnostics}

\subsubsection{Table A1: Summary
Statistics}\label{table-a1-summary-statistics}

\begin{verbatim}
Warning in attr(x, "align"): 'xfun::attr()' is deprecated.
Use 'xfun::attr2()' instead.
See help("Deprecated")
Warning in attr(x, "align"): 'xfun::attr()' is deprecated.
Use 'xfun::attr2()' instead.
See help("Deprecated")
\end{verbatim}

\begin{longtable}[]{@{}ll@{}}
\caption{Data summary}\tabularnewline
\toprule\noalign{}
\endfirsthead
\endhead
\bottomrule\noalign{}
\endlastfoot
Name & Piped data \\
Number of rows & 2222 \\
Number of columns & 18 \\
\_\_\_\_\_\_\_\_\_\_\_\_\_\_\_\_\_\_\_\_\_\_\_ & \\
Column type frequency: & \\
numeric & 18 \\
\_\_\_\_\_\_\_\_\_\_\_\_\_\_\_\_\_\_\_\_\_\_\_\_ & \\
Group variables & None \\
\end{longtable}

\textbf{Variable type: numeric}

\begin{longtable}[]{@{}
  >{\raggedright\arraybackslash}p{(\linewidth - 20\tabcolsep) * \real{0.2093}}
  >{\raggedleft\arraybackslash}p{(\linewidth - 20\tabcolsep) * \real{0.1163}}
  >{\raggedleft\arraybackslash}p{(\linewidth - 20\tabcolsep) * \real{0.1628}}
  >{\raggedleft\arraybackslash}p{(\linewidth - 20\tabcolsep) * \real{0.0698}}
  >{\raggedleft\arraybackslash}p{(\linewidth - 20\tabcolsep) * \real{0.0698}}
  >{\raggedleft\arraybackslash}p{(\linewidth - 20\tabcolsep) * \real{0.0349}}
  >{\raggedleft\arraybackslash}p{(\linewidth - 20\tabcolsep) * \real{0.0698}}
  >{\raggedleft\arraybackslash}p{(\linewidth - 20\tabcolsep) * \real{0.0698}}
  >{\raggedleft\arraybackslash}p{(\linewidth - 20\tabcolsep) * \real{0.0698}}
  >{\raggedleft\arraybackslash}p{(\linewidth - 20\tabcolsep) * \real{0.0581}}
  >{\raggedright\arraybackslash}p{(\linewidth - 20\tabcolsep) * \real{0.0698}}@{}}
\toprule\noalign{}
\begin{minipage}[b]{\linewidth}\raggedright
skim\_variable
\end{minipage} & \begin{minipage}[b]{\linewidth}\raggedleft
n\_missing
\end{minipage} & \begin{minipage}[b]{\linewidth}\raggedleft
complete\_rate
\end{minipage} & \begin{minipage}[b]{\linewidth}\raggedleft
mean
\end{minipage} & \begin{minipage}[b]{\linewidth}\raggedleft
sd
\end{minipage} & \begin{minipage}[b]{\linewidth}\raggedleft
p0
\end{minipage} & \begin{minipage}[b]{\linewidth}\raggedleft
p25
\end{minipage} & \begin{minipage}[b]{\linewidth}\raggedleft
p50
\end{minipage} & \begin{minipage}[b]{\linewidth}\raggedleft
p75
\end{minipage} & \begin{minipage}[b]{\linewidth}\raggedleft
p100
\end{minipage} & \begin{minipage}[b]{\linewidth}\raggedright
hist
\end{minipage} \\
\midrule\noalign{}
\endhead
\bottomrule\noalign{}
\endlastfoot
d\_incppl\_pris\_abc & 33 & 0.99 & 46.13 & 27.96 & 0 & 25.00 & 50.00 &
66.00 & 100 & ▆▆▇▅▃ \\
incarc & 0 & 1.00 & 0.08 & 0.27 & 0 & 0.00 & 0.00 & 0.00 & 1 & ▇▁▁▁▁ \\
knowincarc & 0 & 1.00 & 0.31 & 0.46 & 0 & 0.00 & 0.00 & 1.00 & 1 &
▇▁▁▁▃ \\
vict & 0 & 1.00 & 0.12 & 0.33 & 0 & 0.00 & 0.00 & 0.00 & 1 & ▇▁▁▁▁ \\
desor\_core & 16 & 0.99 & 66.75 & 19.20 & 0 & 52.22 & 66.67 & 81.11 &
100 & ▁▂▇▇▆ \\
fiscal\_ideology & 565 & 0.75 & 2.48 & 1.41 & 0 & 1.00 & 3.00 & 3.00 & 5
& ▇▆▇▃▂ \\
social\_ideology & 565 & 0.75 & 2.26 & 1.48 & 0 & 1.00 & 2.00 & 3.00 & 5
& ▇▆▆▃▂ \\
penalpoint & 190 & 0.91 & 4.08 & 1.81 & 1 & 3.00 & 4.00 & 6.00 & 7 &
▆▅▇▅▇ \\
Republican & 618 & 0.72 & 0.25 & 0.43 & 0 & 0.00 & 0.00 & 0.00 & 1 &
▇▁▁▁▂ \\
Democrat & 618 & 0.72 & 0.39 & 0.49 & 0 & 0.00 & 0.00 & 1.00 & 1 &
▇▁▁▁▅ \\
fire\_rare & 561 & 0.75 & 3.04 & 0.89 & 1 & 2.00 & 3.00 & 4.00 & 4 &
▁▅▁▇▇ \\
fire\_privilege & 561 & 0.75 & 2.14 & 1.02 & 1 & 1.00 & 2.00 & 3.00 & 4
& ▇▇▁▅▃ \\
fire\_angry & 561 & 0.75 & 1.69 & 0.77 & 1 & 1.00 & 2.00 & 2.00 & 4 &
▇▇▁▂▁ \\
fire\_fear & 561 & 0.75 & 3.44 & 0.70 & 1 & 3.00 & 4.00 & 4.00 & 4 &
▁▁▁▆▇ \\
gender & 618 & 0.72 & 0.47 & 0.50 & 0 & 0.00 & 0.00 & 1.00 & 1 &
▇▁▁▁▇ \\
Black & 619 & 0.72 & 0.15 & 0.36 & 0 & 0.00 & 0.00 & 0.00 & 1 & ▇▁▁▁▂ \\
Hispanic & 619 & 0.72 & 0.15 & 0.36 & 0 & 0.00 & 0.00 & 0.00 & 1 &
▇▁▁▁▂ \\
OtherRace & 0 & 1.00 & 0.04 & 0.19 & 0 & 0.00 & 0.00 & 0.00 & 1 &
▇▁▁▁▁ \\
\end{longtable}

Figure A1 shows that perceived deservingness of incarcerated people
centers near 46 on a 0--100 scale (SD ≈ 28), with modest right skew.
Roughly 8 \% of respondents report having been incarcerated, 31 \% know
someone who has, and 12 \% report victimization (Table A2)

\subsubsection{Figure A1. Density of
Dependent}\label{figure-a1.-density-of-dependent}

\begin{verbatim}
# A tibble: 2,189 x 1
   d_incppl_pris_abc
               <dbl>
 1                58
 2               100
 3                83
 4                50
 5                20
 6                 0
 7                84
 8                22
 9                50
10                50
# i 2,179 more rows
\end{verbatim}

\subsubsection{Figure A2. Cross-Tabs of Contact
Types}\label{figure-a2.-cross-tabs-of-contact-types}

\subsection{II. Bivariate Regressions}\label{ii.-bivariate-regressions}

\begin{verbatim}
Warning in attr(x, "align"): 'xfun::attr()' is deprecated.
Use 'xfun::attr2()' instead.
See help("Deprecated")
\end{verbatim}

\begin{verbatim}
Warning in attr(.knitEnv$meta, "knit_meta_id"): 'xfun::attr()' is deprecated.
Use 'xfun::attr2()' instead.
See help("Deprecated")
Warning in attr(.knitEnv$meta, "knit_meta_id"): 'xfun::attr()' is deprecated.
Use 'xfun::attr2()' instead.
See help("Deprecated")
Warning in attr(.knitEnv$meta, "knit_meta_id"): 'xfun::attr()' is deprecated.
Use 'xfun::attr2()' instead.
See help("Deprecated")
Warning in attr(.knitEnv$meta, "knit_meta_id"): 'xfun::attr()' is deprecated.
Use 'xfun::attr2()' instead.
See help("Deprecated")
Warning in attr(.knitEnv$meta, "knit_meta_id"): 'xfun::attr()' is deprecated.
Use 'xfun::attr2()' instead.
See help("Deprecated")
Warning in attr(.knitEnv$meta, "knit_meta_id"): 'xfun::attr()' is deprecated.
Use 'xfun::attr2()' instead.
See help("Deprecated")
Warning in attr(.knitEnv$meta, "knit_meta_id"): 'xfun::attr()' is deprecated.
Use 'xfun::attr2()' instead.
See help("Deprecated")
Warning in attr(.knitEnv$meta, "knit_meta_id"): 'xfun::attr()' is deprecated.
Use 'xfun::attr2()' instead.
See help("Deprecated")
Warning in attr(.knitEnv$meta, "knit_meta_id"): 'xfun::attr()' is deprecated.
Use 'xfun::attr2()' instead.
See help("Deprecated")
Warning in attr(.knitEnv$meta, "knit_meta_id"): 'xfun::attr()' is deprecated.
Use 'xfun::attr2()' instead.
See help("Deprecated")
Warning in attr(.knitEnv$meta, "knit_meta_id"): 'xfun::attr()' is deprecated.
Use 'xfun::attr2()' instead.
See help("Deprecated")
Warning in attr(.knitEnv$meta, "knit_meta_id"): 'xfun::attr()' is deprecated.
Use 'xfun::attr2()' instead.
See help("Deprecated")
Warning in attr(.knitEnv$meta, "knit_meta_id"): 'xfun::attr()' is deprecated.
Use 'xfun::attr2()' instead.
See help("Deprecated")
Warning in attr(.knitEnv$meta, "knit_meta_id"): 'xfun::attr()' is deprecated.
Use 'xfun::attr2()' instead.
See help("Deprecated")
\end{verbatim}

\begin{verbatim}
Warning in attr(x, "align"): 'xfun::attr()' is deprecated.
Use 'xfun::attr2()' instead.
See help("Deprecated")
\end{verbatim}

\begin{verbatim}
Warning in attr(x, "format"): 'xfun::attr()' is deprecated.
Use 'xfun::attr2()' instead.
See help("Deprecated")
\end{verbatim}

\begin{longtable}[t]{lrrrr}
\caption{\label{tab:unnamed-chunk-5}Table 1. Coefficients for M2 – Incarcerated}\\
\toprule
term & estimate & std.error & statistic & p.value\\
\midrule
(Intercept) & 4.7544258 & 6.1042170 & 0.7788756 & 0.4361710\\
incarc & 8.6976050 & 2.4149822 & 3.6015193 & 0.0003263\\
Republican & -0.4663814 & 1.6684804 & -0.2795247 & 0.7798792\\
Democrat & -0.7256735 & 1.4639820 & -0.4956847 & 0.6201867\\
fiscal\_ideology & 0.2127949 & 0.8273743 & 0.2571930 & 0.7970637\\
\addlinespace
social\_ideology & -1.9809023 & 0.8482607 & -2.3352516 & 0.0196564\\
penalpoint & -1.2055685 & 0.3478478 & -3.4657933 & 0.0005430\\
desor\_core & 0.7376188 & 0.0374672 & 19.6870629 & 0.0000000\\
fire\_rare & 0.4964744 & 0.8534110 & 0.5817530 & 0.5608171\\
fire\_privilege & -0.2067427 & 0.8051144 & -0.2567867 & 0.7973773\\
\addlinespace
fire\_angry & 0.6873672 & 0.9038487 & 0.7604892 & 0.4470771\\
fire\_fear & -0.2047978 & 0.9891034 & -0.2070540 & 0.8359947\\
gender & -1.2442991 & 1.1557388 & -1.0766265 & 0.2818135\\
Black & -0.3019164 & 1.8744224 & -0.1610717 & 0.8720578\\
Hispanic & -3.0669029 & 1.7254155 & -1.7774866 & 0.0756829\\
\addlinespace
OtherRace & -1.1626656 & 2.3454755 & -0.4957057 & 0.6201718\\
\bottomrule
\end{longtable}

\begin{verbatim}
Warning in attr(.knitEnv$meta, "knit_meta_id"): 'xfun::attr()' is deprecated.
Use 'xfun::attr2()' instead.
See help("Deprecated")
\end{verbatim}

\begin{verbatim}
Warning in attr(.knitEnv$meta, "knit_meta_id"): 'xfun::attr()' is deprecated.
Use 'xfun::attr2()' instead.
See help("Deprecated")
Warning in attr(.knitEnv$meta, "knit_meta_id"): 'xfun::attr()' is deprecated.
Use 'xfun::attr2()' instead.
See help("Deprecated")
Warning in attr(.knitEnv$meta, "knit_meta_id"): 'xfun::attr()' is deprecated.
Use 'xfun::attr2()' instead.
See help("Deprecated")
Warning in attr(.knitEnv$meta, "knit_meta_id"): 'xfun::attr()' is deprecated.
Use 'xfun::attr2()' instead.
See help("Deprecated")
Warning in attr(.knitEnv$meta, "knit_meta_id"): 'xfun::attr()' is deprecated.
Use 'xfun::attr2()' instead.
See help("Deprecated")
Warning in attr(.knitEnv$meta, "knit_meta_id"): 'xfun::attr()' is deprecated.
Use 'xfun::attr2()' instead.
See help("Deprecated")
Warning in attr(.knitEnv$meta, "knit_meta_id"): 'xfun::attr()' is deprecated.
Use 'xfun::attr2()' instead.
See help("Deprecated")
Warning in attr(.knitEnv$meta, "knit_meta_id"): 'xfun::attr()' is deprecated.
Use 'xfun::attr2()' instead.
See help("Deprecated")
Warning in attr(.knitEnv$meta, "knit_meta_id"): 'xfun::attr()' is deprecated.
Use 'xfun::attr2()' instead.
See help("Deprecated")
Warning in attr(.knitEnv$meta, "knit_meta_id"): 'xfun::attr()' is deprecated.
Use 'xfun::attr2()' instead.
See help("Deprecated")
Warning in attr(.knitEnv$meta, "knit_meta_id"): 'xfun::attr()' is deprecated.
Use 'xfun::attr2()' instead.
See help("Deprecated")
Warning in attr(.knitEnv$meta, "knit_meta_id"): 'xfun::attr()' is deprecated.
Use 'xfun::attr2()' instead.
See help("Deprecated")
Warning in attr(.knitEnv$meta, "knit_meta_id"): 'xfun::attr()' is deprecated.
Use 'xfun::attr2()' instead.
See help("Deprecated")
\end{verbatim}

\begin{verbatim}
Warning in attr(x, "align"): 'xfun::attr()' is deprecated.
Use 'xfun::attr2()' instead.
See help("Deprecated")
\end{verbatim}

\begin{verbatim}
Warning in attr(x, "format"): 'xfun::attr()' is deprecated.
Use 'xfun::attr2()' instead.
See help("Deprecated")
\end{verbatim}

\begin{table}
\centering
\caption{\label{tab:unnamed-chunk-5}Table 1. Coefficients for M2 – Victimized}
\centering
\begin{tabular}[t]{l|r|r|r|r}
\hline
term & estimate & std.error & statistic & p.value\\
\hline
(Intercept) & 4.0757569 & 6.1525875 & 0.6624460 & 0.5077832\\
\hline
vict & 1.8311316 & 2.0491020 & 0.8936264 & 0.3716596\\
\hline
Republican & -0.2417254 & 1.6813713 & -0.1437668 & 0.8857032\\
\hline
Democrat & -0.8120548 & 1.4681640 & -0.5531091 & 0.5802679\\
\hline
fiscal\_ideology & 0.1746308 & 0.8356793 & 0.2089687 & 0.8345000\\
\hline
social\_ideology & -1.8930024 & 0.8491248 & -2.2293572 & 0.0259320\\
\hline
penalpoint & -1.1615853 & 0.3475277 & -3.3424249 & 0.0008501\\
\hline
desor\_core & 0.7400902 & 0.0375803 & 19.6935530 & 0.0000000\\
\hline
fire\_rare & 0.5040169 & 0.8629299 & 0.5840763 & 0.5592534\\
\hline
fire\_privilege & -0.2149014 & 0.8134293 & -0.2641918 & 0.7916670\\
\hline
fire\_angry & 0.6428042 & 0.9022890 & 0.7124150 & 0.4763143\\
\hline
fire\_fear & -0.0883885 & 0.9876795 & -0.0894911 & 0.9287031\\
\hline
gender & -0.9467339 & 1.1612650 & -0.8152609 & 0.4150474\\
\hline
Black & 0.1874133 & 1.8721190 & 0.1001076 & 0.9202718\\
\hline
Hispanic & -2.9739921 & 1.7405186 & -1.7086816 & 0.0877087\\
\hline
OtherRace & -0.8587638 & 2.3653051 & -0.3630668 & 0.7166041\\
\hline
\end{tabular}
\end{table}

\begin{verbatim}
Warning in attr(.knitEnv$meta, "knit_meta_id"): 'xfun::attr()' is deprecated.
Use 'xfun::attr2()' instead.
See help("Deprecated")
\end{verbatim}

\begin{verbatim}
Warning in attr(.knitEnv$meta, "knit_meta_id"): 'xfun::attr()' is deprecated.
Use 'xfun::attr2()' instead.
See help("Deprecated")
Warning in attr(.knitEnv$meta, "knit_meta_id"): 'xfun::attr()' is deprecated.
Use 'xfun::attr2()' instead.
See help("Deprecated")
Warning in attr(.knitEnv$meta, "knit_meta_id"): 'xfun::attr()' is deprecated.
Use 'xfun::attr2()' instead.
See help("Deprecated")
Warning in attr(.knitEnv$meta, "knit_meta_id"): 'xfun::attr()' is deprecated.
Use 'xfun::attr2()' instead.
See help("Deprecated")
Warning in attr(.knitEnv$meta, "knit_meta_id"): 'xfun::attr()' is deprecated.
Use 'xfun::attr2()' instead.
See help("Deprecated")
Warning in attr(.knitEnv$meta, "knit_meta_id"): 'xfun::attr()' is deprecated.
Use 'xfun::attr2()' instead.
See help("Deprecated")
Warning in attr(.knitEnv$meta, "knit_meta_id"): 'xfun::attr()' is deprecated.
Use 'xfun::attr2()' instead.
See help("Deprecated")
Warning in attr(.knitEnv$meta, "knit_meta_id"): 'xfun::attr()' is deprecated.
Use 'xfun::attr2()' instead.
See help("Deprecated")
Warning in attr(.knitEnv$meta, "knit_meta_id"): 'xfun::attr()' is deprecated.
Use 'xfun::attr2()' instead.
See help("Deprecated")
Warning in attr(.knitEnv$meta, "knit_meta_id"): 'xfun::attr()' is deprecated.
Use 'xfun::attr2()' instead.
See help("Deprecated")
Warning in attr(.knitEnv$meta, "knit_meta_id"): 'xfun::attr()' is deprecated.
Use 'xfun::attr2()' instead.
See help("Deprecated")
Warning in attr(.knitEnv$meta, "knit_meta_id"): 'xfun::attr()' is deprecated.
Use 'xfun::attr2()' instead.
See help("Deprecated")
Warning in attr(.knitEnv$meta, "knit_meta_id"): 'xfun::attr()' is deprecated.
Use 'xfun::attr2()' instead.
See help("Deprecated")
\end{verbatim}

\begin{verbatim}
Warning in attr(x, "align"): 'xfun::attr()' is deprecated.
Use 'xfun::attr2()' instead.
See help("Deprecated")
\end{verbatim}

\begin{verbatim}
Warning in attr(x, "format"): 'xfun::attr()' is deprecated.
Use 'xfun::attr2()' instead.
See help("Deprecated")
\end{verbatim}

\begin{table}
\centering
\caption{\label{tab:unnamed-chunk-5}Table 1. Coefficients for M2 – Victimized}
\centering
\begin{tabular}[t]{l|r|r|r|r}
\hline
term & estimate & std.error & statistic & p.value\\
\hline
(Intercept) & 3.2377490 & 6.1291670 & 0.5282527 & 0.5973991\\
\hline
knowincarc & 5.3546148 & 1.2577090 & 4.2574353 & 0.0000219\\
\hline
Republican & -0.0704740 & 1.6726961 & -0.0421320 & 0.9663989\\
\hline
Democrat & -0.7769671 & 1.4644549 & -0.5305504 & 0.5958058\\
\hline
fiscal\_ideology & 0.0352355 & 0.8315015 & 0.0423758 & 0.9662046\\
\hline
social\_ideology & -1.9609376 & 0.8476605 & -2.3133526 & 0.0208327\\
\hline
penalpoint & -1.1745083 & 0.3455765 & -3.3986928 & 0.0006941\\
\hline
desor\_core & 0.7347021 & 0.0376058 & 19.5369498 & 0.0000000\\
\hline
fire\_rare & 0.4148532 & 0.8536177 & 0.4859942 & 0.6270395\\
\hline
fire\_privilege & -0.0513998 & 0.8100706 & -0.0634510 & 0.9494155\\
\hline
fire\_angry & 0.6068086 & 0.8962396 & 0.6770607 & 0.4984678\\
\hline
fire\_fear & 0.0115221 & 0.9921904 & 0.0116127 & 0.9907361\\
\hline
gender & -0.9481606 & 1.1552792 & -0.8207198 & 0.4119312\\
\hline
Black & -0.1224413 & 1.8729459 & -0.0653736 & 0.9478849\\
\hline
Hispanic & -3.3836311 & 1.7201913 & -1.9670086 & 0.0493591\\
\hline
OtherRace & -0.4743038 & 2.3410358 & -0.2026042 & 0.8394708\\
\hline
\end{tabular}
\end{table}

In bivariate models, desor\_core alone explains about 30 \% of the
variance in deservingness, dwarfing all other predictors (Table B1).
Ideological measures account for 7--8 \% of variance; punitiveness about
3--4 \%. Carceral‐contact variables each explain \textless{} 1 \% of
variance individually.

\subsubsection{Table B1: Adjusted R² and AIC for each predictor (sorted
by Adj.
R²).}\label{table-b1-adjusted-ruxb2-and-aic-for-each-predictor-sorted-by-adj.-ruxb2.}

\subsection{III. Full Models by Contact
Type}\label{iii.-full-models-by-contact-type}

\subsubsection{A. M2 - Incarcerated}\label{a.-m2---incarcerated}

\subsubsection{Table 1. Coefficients for M2 -
Incarcerated}\label{table-1.-coefficients-for-m2---incarcerated}

\subsubsection{Table 2. Semi-partial R2 -
Incarcerated}\label{table-2.-semi-partial-r2---incarcerated}

\subsubsection{Table 3. Standardized betas -
Incarcerated}\label{table-3.-standardized-betas---incarcerated}

\subsubsection{A. M2 - Victimized}\label{a.-m2---victimized}

\subsubsection{Table 1. Coefficients for M2 -
Victimized}\label{table-1.-coefficients-for-m2---victimized}

\subsubsection{Table 2. Semi-partial R2 -
Victimized}\label{table-2.-semi-partial-r2---victimized}

\subsubsection{Table 3. Standardized betas -
Victimized}\label{table-3.-standardized-betas---victimized}

\subsubsection{A. M2 - Proximate}\label{a.-m2---proximate}

\subsubsection{Table 1. Coefficients for M2 -
Proximate}\label{table-1.-coefficients-for-m2---proximate}

\subsubsection{Table 2. Semi-partial R2 -
Proximate}\label{table-2.-semi-partial-r2---proximate}

\subsubsection{Table 3. Standardized betas -
Proximate}\label{table-3.-standardized-betas---proximate}

\subsection{IV. Cross-Model
Comparison}\label{iv.-cross-model-comparison}

Across models (Tables 1--12), generalized deservingness orientations
remain the most powerful predictor of perceived deservingness of
incarcerated people.

A 10-point increase in desor\_core predicts roughly a 7.4-point higher
deservingness score.

The standardized coefficient (β ≈ 0.51) indicates a half-SD shift in the
outcome for each SD increase in desor\_core.

incarc (+8.7, p \textless{} .001) and knowincarc (+5.4, p \textless{}
.001) also raise perceived deservingness, but their standardized effects
are modest (β ≈ 0.08 and β ≈ 0.06, respectively).

Victimization is positive but not significant (β ≈ 0.02).

More socially conservative and punitive respondents view incarcerated
people as less deserving (social\_ideology ≈ −0.11 β; penalpoint ≈ −0.08
β).

Partisan identification, FIRE items, and demographics contribute
negligibly.

Variance Decomposition: Semi-partial R² values underscore the hierarchy
of explanatory power (Figure C1). General deservingness beliefs
(desor\_core) uniquely account for about 23 \% of explained variance in
deservingness judgments, whereas experiential contact adds only 0.6--1.1
\%, and ideology or punitiveness add around 0.7 \% each.

These results suggest that attitudinal orientations about help and
deservingness dominate, but personal contact with the carceral system
still exerts an independent, substantively meaningful effect. Direct or
proximate experience corresponds to more generous views of incarcerated
people, while victimization does not.

\section{written oct 31}\label{written-oct-31}

In a series of bivariate regressions, I examined how much variation in
deservingness perceptions of incarcerated people
(\texttt{d\_incppl\_pris\_abc}) each control variable accounted for.
Adjusted R² values indicate the proportion of total outcome variance
explained by each predictor. For example, baseline help orientations
(\texttt{desor\_core}) alone explain roughly 30\% of the variance in
deservingness -- meaning that people's differences in perceived
deservingness align strongly with their differences in support levels.
Other predictors, such as ideology or partisanship, each explain smaller
portions of the variance (roughly 3--8\%), while carceral impact
explains little by itself.

In the full model, experiencing carceral impact is associated with a
\textbf{4.6-point} increase in the policy support index (0--100),
controlling for ideology, party ID, and deservingness. By contrast, a
10-point increase in the \texttt{desor\_core} index predicts about a
\textbf{7.3-point} increase. Standardized coefficients and semi-partial
R² confirm that \texttt{desor\_core} accounts for more of the outcome's
variance than \texttt{impact} does uniquely; however \texttt{impact}
still has an independent, statistically significant association. This
pattern suggests that while personal impact matters, core deservingness
beliefs explain a larger portion of cross-person variation in support.

When all variables are standardized, \emph{desor\_core} has by far the
largest standardized coefficient (β = 0.51), indicating that respondents
who rate carceral citizens as more deserving tend to express much
stronger support for prison-help policies. A one--standard deviation
increase in \emph{desor\_core} corresponds to roughly a half--standard
deviation increase in support.

The coefficient for \emph{impact} (β = 0.08) remains positive and
statistically significant, suggesting that direct or indirect experience
with the carceral system is associated with higher policy support,
though the magnitude is relatively modest compared with attitudinal
factors.

\emph{Social ideology} (β = −0.11) and \emph{punitiveness} (β = −0.07)
show smaller negative relationships, while partisan identifiers and FIRE
items contribute little once these factors are controlled.

\textbf{Incarceration on Deservingness}

When we include both experiential exposure (\texttt{incarc}) and the
attitudinal control reflecting general orientation toward help-needing
groups (\texttt{desor\_core}) in the same model, both predictors are
statistically significant. The semi-partial R² values reveal that
\texttt{desor\_core} uniquely accounts for approximately \textbf{23\% of
the variance} in policy support, whereas \texttt{incarc} explains less
than \textbf{1\% of unique variance}. This indicates that generalized
perceptions of deservingness are the dominant driver of support, while
personal experience with incarceration contributes a smaller, though
still meaningful, independent effect.

Because deservingness orientations (\texttt{desor\_core}) strongly
predict views on incarcerated people and may correlate with experience
of incarceration (\texttt{incarc}), we include \texttt{desor\_core} as a
control in all models. Controlling for \texttt{desor\_core} allows us to
estimate the independent association of incarceration experience with
policy support, net of respondents' attitudinal differences. The
semi-partial R² results highlight the relative explanatory contributions
of each predictor, confirming that attitudes toward carceral citizens
are far more influential than demographic or ideological controls.

These results demonstrate that while attitudes about deservingness
dominate variance in policy support, experiential exposure to the
criminal legal system has a measurable, independent effect --- which
aligns with our experimental interest in whether incarceration shapes
perceptions of carceral citizens (\texttt{d\_incppl\_pris\_abc}).

When we include both experiential exposure (\texttt{incarc}) and the
control for generalized deservingness attitudes (\texttt{desor\_core})
in the same model predicting policy support, both predictors are
statistically significant. The unstandardized coefficient for
\texttt{incarc} indicates that respondents who have experienced
incarceration score roughly \textbf{8.7 points higher} on the policy
support scale (0--100), holding other variables constant. By comparison,
a one-unit increase in \texttt{desor\_core} is associated with a
\textbf{0.74-point increase}, though the 0--100 scale of this composite
makes direct comparisons challenging.

Standardized coefficients place all predictors on the same scale and
clarify relative influence: a \textbf{1 SD increase in
\texttt{desor\_core} corresponds to a 0.51 SD increase} in policy
support, whereas a change from 0 to 1 in \texttt{incarc} corresponds to
only a \textbf{0.079 SD increase}. Similarly, semi-partial R² values
show that \texttt{desor\_core} uniquely explains
\textbf{\textasciitilde23\% of the variance}, while \texttt{incarc}
explains less than \textbf{1\%}, confirming that generalized attitudes
about deservingness are the dominant driver of support.

Including \texttt{desor\_core} as a control allows us to isolate the
independent association of incarceration experience with policy support,
net of respondents' broader attitudinal dispositions. Taken together,
these results indicate that while attitudes toward deservingness
strongly predict policy support, personal experience with the criminal
legal system exerts a smaller but meaningful effect --- which aligns
with our experimental interest in whether incarceration shapes
perceptions of carceral citizens (\texttt{d\_incppl\_pris\_abc}).

\section{Results: M3 \& M4 -- REWRITE AFTER HAVING FIXED
SCALES}\label{results-m3-m4-rewrite-after-having-fixed-scales}

My second core hypothesis (\textbf{H2)} is that more generous
perceptions of carceral citizens' deservingness will predict greater and
more consistent support for assistance-oriented criminal legal policies
and reduced support for punitive criminal legal policies.

\subsection{Model 3: Help-Oriented Policy
Support}\label{model-3-help-oriented-policy-support}

After estimating the outcome model for the help-policy index, I find
that perceptions of incarcerated people's deservingness are a consistent
and statistically significant predictor of support for
assistance-oriented criminal legal policies. Each one-point increase on
the 0--100 deservingness scale corresponds to a 0.005-point increase on
the 1--4 policy-support scale (p \textless{} .001). Although
substantively modest, this relationship is highly robust and in the
expected direction---respondents who view incarcerated people as more
deserving are more supportive of policies that expand rehabilitation,
education, and family contact.

In this specification, respondents with any form of carceral
contact---whether through direct incarceration, knowing someone who has
been incarcerated, or victimization---express significantly greater
support for help-oriented policies (b = 0.17, p \textless{} .001). This
finding departs from earlier models where the contact effect attenuated
once attitudes were controlled, suggesting that contact and
deservingness perceptions now operate as complementary rather than
competing predictors. The pattern is consistent with the possibility
that contact influences help-policy support both directly and indirectly
through its effect on deservingness, a mechanism tested explicitly in
subsequent mediation models (M7--M8).

Among covariates, a more punitive view of the purpose of prisons
(penalpoint) is associated with lower support for help policies (p
\textless{} .001), and social conservatism also predicts reduced support
(p \textless{} .001). General generosity toward help-needing groups
(\texttt{desor\_core}) likewise increases help-policy support (p =
.001), suggesting that both domain-specific and broad moral orientations
matter for rehabilitative policy preferences. Racial attitudes exert
expected effects: disagreement that racial problems are rare and
disagreement with fear of other races predict greater support for help
policies, while denying White privilege and expressing less anger about
racism predict lower support. The model explains roughly 37 percent of
the variance in help-policy support (Adj. R² = .37), a substantial
improvement from earlier specifications and consistent with expectations
that attitudinal factors---particularly deservingness
perceptions---dominate these evaluations.

\subsection{Model 4: Opposition to Punitive
Policies}\label{model-4-opposition-to-punitive-policies}

Turning to opposition to punitive policies, the relationship between
perceived deservingness of incarcerated people and policy preferences
remains strong and in the expected direction. Each one-point increase in
deservingness corresponds to a 0.005-point increase in opposition to
punitive policies (p \textless{} .001). Thus, respondents who see
incarcerated people as more deserving are less supportive of punitive
measures such as the death penalty or life without parole. Individuals
with carceral contact also express greater opposition to punitive
policies (b = 0.08, p = .03), indicating that lived experience with the
criminal legal system corresponds to less punitive attitudes.

Fiscal and social conservatism are both negatively associated with
opposition to punitive policies, while a punitive view of the prison's
purpose (penalpoint) again predicts greater support for punitive
measures (p \textless{} .001). Among the FIRE items, acknowledging White
privilege (p = .007) and expressing anger about racism (p = .019) each
correspond to greater opposition to punitive policies, whereas racial
fear predicts stronger punitiveness (p = .024). Other demographic
variables are statistically indistinguishable from zero. The model
explains roughly 16 percent of the variance in opposition to punitive
policies (Adj. R² = .16), indicating that deservingness and punitive
orientation remain the principal attitudinal anchors of criminal legal
policy opinion.

\subsection{Distinguishing General and Carceral
Deservingness}\label{distinguishing-general-and-carceral-deservingness}

Because generosity toward incarcerated people may reflect a broader
prosocial orientation, I include a control for respondents'
\textbf{general deservingness orientation} (\texttt{desor\_core}),
measured as the mean of deservingness ratings for help-needing groups
such as welfare recipients, the homeless, and immigrants. This variable
captures an underlying disposition toward helping others. Diagnostic
tests show that \texttt{desor\_core} and incarcerated-group
deservingness are moderately correlated (r = .55), but the shared
variance is conceptually meaningful rather than problematic. In the
help-policy model, \texttt{desor\_core} contributes additional
explanatory power (ΔAdj. R² ≈ .05), whereas in the punitive-policy
model, its contribution is negligible (ΔAdj. R² ≈ .001). These patterns
indicate that while general generosity reinforces support for
assistance-oriented carceral policies, opposition to punitive policies
is driven primarily by attitudes specific to the incarcerated
population.

START OLD old M3M4

M3

After estimating the outcome model for the help policy index, I find
that perceptions of incarcerated people's deservingness are a consistent
and statistically significant predictor of support for
assistance-oriented criminal legal policies. Each one-point increase on
the 0--100 deservingness scale corresponds to a 0.004-point increase on
the 1--5 policy-support scale (p \textless{} .001). Although modest in
magnitude, this relationship is robust and in the theoretically expected
direction.

Contrary to my expectations, carceral contact itself---whether through
direct experience, knowing someone incarcerated, or victimization---has
no statistically discernible effect on support for help policies once
attitudes and ideology are accounted for. Although direct carceral
contact does not predict help-policy support once deservingness is
included, this attenuation is consistent with the possibility that
contact's influence operates indirectly through shifts in deservingness
perceptions---a question tested directly in later mediation models
(M7--M8).

Both Republican and Democratic identifiers express lower support for
such policies relative to independents, and a more punitive orientation
toward the purpose of prisons is strongly associated with reduced
support (p \textless{} .001). Social and fiscal ideology are
statistically indistinguishable from zero in this model, suggesting that
specific penal and deservingness attitudes are more relevant than broad
ideological self-placement. The model explains approximately 13 percent
of the variance in policy support (Adj. R² = .13), indicating that while
the predictors collectively matter, most of the variance lies beyond the
attitudinal and demographic factors captured here.

M4

Turning to the model predicting support for punitive policies, the
relationship between perceived deservingness of incarcerated people and
policy preferences remains strong and in the expected negative
direction. Each one-point increase in deservingness corresponds to a
0.005-point decrease in support for punitive policies on the 1--5 scale
(p \textless{} .001), indicating that respondents who view incarcerated
people as more deserving are less supportive of punitive measures. In
contrast to the help-policy model, individuals with any carceral contact
report higher punitiveness (b = 0.095, p = .019), suggesting a modest
tendency toward more punitive views among those directly or indirectly
affected by the system. Fiscal and social conservatism are both
negatively associated with support for punitive reforms, while a
punitive orientation toward the purpose of prisons (penalpoint) again
strongly predicts favoring tougher policies (p \textless{} .001). Among
the racial-attitudes measures, believing that White privilege exists
(fire\_privilege) and feeling anger about racism (fire\_angry) each
relate to lower punitiveness, whereas fear of other races shows a
marginal positive relationship (p = .051). Other demographic variables
are statistically indistinguishable from zero. The model explains
roughly 17 percent of the variance in punitive policy preferences (Adj.
R² = .16), underscoring that carceral-specific deservingness and
punitive orientation remain the primary attitudinal foundations of
support for these policies.

Because attitudes toward incarcerated people may partly reflect broader
social attitudes towards help-needing groups, I include a control for
general deservingness orientation to ensure the observed effects are not
artifacts of a general helping disposition. I isolate the unique
relationship between respondents' perceptions of incarcerated people and
their policy preferences in both models using this control(desor\_core),
constructed as the mean of respondents' deservingness ratings for
conventionally ``help-needing'' groups (e.g., the homeless, welfare
recipients, and immigrants). This variable captures an underlying
generosity disposition toward social assistance targets, allowing
estimation of whether the effect of incarcerated-group deservingness
persists above and beyond general helping attitudes. Diagnostic tests
indicate that desor\_core and incarcerated-group deservingness are
moderately correlated (r = .55), but desor\_core contributes negligible
unique variance once the specific deservingness measure is included
(semi-partial R² \textless{} .001; ΔAdj R² ≈ .005). These results
suggest that while general deservingness orientation and
incarcerated-group deservingness are closely related, the association
between the latter and policy support reflects attitudes specific to the
carceral context rather than a broad disposition toward help-needing
groups.

\textbf{\emph{end old M3M4}}

\textbf{\emph{M3M4 By Impact}}

\section{Results: M5 \& 6}\label{results-m5-6}

My third core hypothesis (H3) is that perceptions of incarcerated
people's deservingness mediate the relationship between carceral contact
and support for criminal legal policies. Specifically, I expect that
direct or indirect experience with the carceral system influences
deservingness perceptions, which in turn shape policy preferences.
Before turning to the mediation tests, Models 5 and 6 first evaluate
whether contact moderates the effect of deservingness on policy
support---that is, whether deservingness operates differently across
those with and without criminal-legal experience.

Model 5: Deservingness × Contact on Help-Oriented Policy To assess
whether the relationship between perceptions of incarcerated people's
deservingness and support for assistance-oriented carceral policies
varies by personal contact with the criminal legal system, I estimated
an interaction model in which deservingness (0--100 scale) and a binary
indicator of carceral contact jointly predict the prisonhelp index (1--5
scale). Both variables were mean-centered to facilitate interpretation,
such that the main effect of deservingness represents the slope for
non-impacted respondents at the average level of deservingness.

{[}add latex formula{]}

After correcting the coding of help-policy items, both deservingness and
contact exert independent, positive effects on support for
assistance-oriented prison policies. Deservingness remains a strong
predictor (\$b = 0.0054\$, SE = 0.0007, \$p \textless{} .001\$): a
10-point increase in deservingness corresponds to roughly a 0.05-point
rise on the five-point prisonhelp index. Respondents who have
experienced or witnessed criminal-legal contact express markedly higher
baseline support for such policies (\$b = 0.17\$, SE = 0.03, \$p
\textless{} .001\$), even when controlling for partisanship, ideology,
punitiveness, and generalized deservingness.

The interaction between deservingness and contact is small and
statistically indistinguishable from zero (\$b = --0.0014\$, \$p =
.16\$), indicating that the positive effect of deservingness is
comparable across groups. Both impacted and non-impacted respondents
increase in policy support at roughly the same rate as deservingness
rises. Simple-slope estimates confirm that the relationship is slightly
stronger for non-impacted respondents (+0.005 vs.~+0.004 per
deservingness point), but the difference is not statistically
meaningful. Predicted values show parallel upward-sloping lines: support
increases from about 2.7 to 3.3 as deservingness rises, with impacted
respondents consistently about 0.17 points more supportive across the
range. These results indicate that contact raises baseline support but
does not moderate the influence of deservingness.

Among the controls, conservative social ideology and a punitive view of
incarceration's purpose strongly depress help-policy support, while
broader prosocial orientations (desor\_core, fire\_rare) enhance it. The
full model explains about 37 \% of the variance in help-policy support
(Adj. \$R\^{}2 = 0.37\$), more than double the fit of the earlier
specification.

Model 6: Deservingness × Contact on Opposition to Punitive Policy

Model 6 evaluates whether the association between perceptions of
incarcerated people's deservingness and opposition to punitive carceral
policies varies by contact with the criminal legal system. Following the
corrected coding of the punitive index (higher values = greater
opposition to punitive policy), both deservingness and contact exhibit
independent, positive effects. Deservingness predicts significantly
greater opposition to punitive policies (\$b = 0.0052\$, SE = 0.0009,
\$p \textless{} .001\$): each 10-point increase in perceived
deservingness corresponds to roughly a 0.05-point rise on the five-point
prisonpen index, indicating a less punitive orientation. Respondents
with carceral contact are modestly less punitive overall (\$b = 0.076\$,
SE = 0.035, \$p = .03\$), though the effect size is smaller than for
help-policy support.

The interaction between deservingness and contact is negligible (\$b =
--0.0002\$, \$p = .89\$), demonstrating that the de-punitive effect of
deservingness operates similarly across groups. Simple-slope analyses
confirm this pattern: among non-impacted respondents, a one-point
increase in deservingness predicts 0.0052 greater opposition to punitive
policy (SE = 0.0009, \$p \textless{} .001\$), and among impacted
respondents 0.0050 (SE = 0.0011, \$p \textless{} .001\$). Predicted
values range from roughly 1.9 to 2.5 across the observed deservingness
distribution, with lines for impacted and non-impacted respondents
nearly parallel and separated by about 0.07 points---contact elevates
overall decarceral sentiment but does not alter the slope of
deservingness.

Conservative fiscal and social ideology continue to predict greater
punitiveness, while a punitive conception of the prison's purpose
(penalpoint) strongly decreases opposition to punitive policies. The
FIRE items again differentiate affective dimensions of racial attitudes:
anger increases punitiveness, whereas privilege and fear reduce it
modestly. The model explains 16.8 \% of variance in punitive-policy
opposition (Adj. \$R\^{}2 = 0.16\$).

Joint Interpretation of Models 5 and 6 Across both help- and
punitive-policy domains, deservingness and contact exhibit consistent,
independent effects. Individuals who perceive incarcerated people as
more deserving are simultaneously more supportive of rehabilitative
policies and more opposed to punitive ones, and these effects are
statistically indistinguishable across contact groups. Personal
experience with the carceral system primarily shifts the baseline level
of policy support upward rather than conditioning how deservingness
translates into attitudes.

Semi-partial R² comparisons reinforce this pattern. In the help-policy
model, deservingness accounts for 3.6 \% of unique variance compared to
2.3 \% for contact; in the punitive-opposition model, deservingness
explains 2.2 \% and contact only 0.3 \%. Thus, moral evaluations of
incarcerated individuals explain substantially more of the variance in
policy preferences than does lived experience itself. Taken together,
Models 5 and 6 provide strong evidence that perceptions of deservingness
are the dominant attitudinal mechanism shaping citizens' responses to
carceral policy, while personal contact functions mainly as a general
decarceral disposition.

\section{Results M7 M8 (Mediation}\label{results-m7-m8-mediation}

Causal mediation analyses indicate that deservingness perceptions
partially mediate the effect of carceral contact on policy attitudes.
For help-oriented policies, the indirect effect of contact through
deservingness (ACME = 0.022, 95\% CI {[}0.010, 0.036{]}) accounts for
roughly 12\% of the total effect. For punitive policies, the indirect
pathway is similarly significant (ACME = 0.024, 95\% CI {[}0.010,
0.040{]}), explaining approximately 24\% of the total effect. In both
domains, contact also retains a positive direct effect, suggesting that
while deservingness plays a meaningful role, experiential exposure
influences policy orientations through additional mechanisms as well.




\end{document}
